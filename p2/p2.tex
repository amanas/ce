
% Default to the notebook output style

    


% Inherit from the specified cell style.




    
\documentclass[11pt]{article}

    
    
    \usepackage[T1]{fontenc}
    % Nicer default font than Computer Modern for most use cases
    \usepackage{palatino}

    % Basic figure setup, for now with no caption control since it's done
    % automatically by Pandoc (which extracts ![](path) syntax from Markdown).
    \usepackage{graphicx}
    % We will generate all images so they have a width \maxwidth. This means
    % that they will get their normal width if they fit onto the page, but
    % are scaled down if they would overflow the margins.
    \makeatletter
    \def\maxwidth{\ifdim\Gin@nat@width>\linewidth\linewidth
    \else\Gin@nat@width\fi}
    \makeatother
    \let\Oldincludegraphics\includegraphics
    % Set max figure width to be 80% of text width, for now hardcoded.
    \renewcommand{\includegraphics}[1]{\Oldincludegraphics[width=.8\maxwidth]{#1}}
    % Ensure that by default, figures have no caption (until we provide a
    % proper Figure object with a Caption API and a way to capture that
    % in the conversion process - todo).
    \usepackage{caption}
    \DeclareCaptionLabelFormat{nolabel}{}
    \captionsetup{labelformat=nolabel}

    \usepackage{adjustbox} % Used to constrain images to a maximum size 
    \usepackage{xcolor} % Allow colors to be defined
    \usepackage{enumerate} % Needed for markdown enumerations to work
    \usepackage{geometry} % Used to adjust the document margins
    \usepackage{amsmath} % Equations
    \usepackage{amssymb} % Equations
    \usepackage{textcomp} % defines textquotesingle
    % Hack from http://tex.stackexchange.com/a/47451/13684:
    \AtBeginDocument{%
        \def\PYZsq{\textquotesingle}% Upright quotes in Pygmentized code
    }
    \usepackage{upquote} % Upright quotes for verbatim code
    \usepackage{eurosym} % defines \euro
    \usepackage[mathletters]{ucs} % Extended unicode (utf-8) support
    \usepackage[utf8x]{inputenc} % Allow utf-8 characters in the tex document
    \usepackage{fancyvrb} % verbatim replacement that allows latex
    \usepackage{grffile} % extends the file name processing of package graphics 
                         % to support a larger range 
    % The hyperref package gives us a pdf with properly built
    % internal navigation ('pdf bookmarks' for the table of contents,
    % internal cross-reference links, web links for URLs, etc.)
    \usepackage{hyperref}
    \usepackage{longtable} % longtable support required by pandoc >1.10
    \usepackage{booktabs}  % table support for pandoc > 1.12.2
    \usepackage[normalem]{ulem} % ulem is needed to support strikethroughs (\sout)
                                % normalem makes italics be italics, not underlines
    \usepackage{fancyhdr}
    \usepackage{parskip}

    

    
    
    % Colors for the hyperref package
    \definecolor{urlcolor}{rgb}{0,.145,.698}
    \definecolor{linkcolor}{rgb}{.71,0.21,0.01}
    \definecolor{citecolor}{rgb}{.12,.54,.11}

    % ANSI colors
    \definecolor{ansi-black}{HTML}{3E424D}
    \definecolor{ansi-black-intense}{HTML}{282C36}
    \definecolor{ansi-red}{HTML}{E75C58}
    \definecolor{ansi-red-intense}{HTML}{B22B31}
    \definecolor{ansi-green}{HTML}{00A250}
    \definecolor{ansi-green-intense}{HTML}{007427}
    \definecolor{ansi-yellow}{HTML}{DDB62B}
    \definecolor{ansi-yellow-intense}{HTML}{B27D12}
    \definecolor{ansi-blue}{HTML}{208FFB}
    \definecolor{ansi-blue-intense}{HTML}{0065CA}
    \definecolor{ansi-magenta}{HTML}{D160C4}
    \definecolor{ansi-magenta-intense}{HTML}{A03196}
    \definecolor{ansi-cyan}{HTML}{60C6C8}
    \definecolor{ansi-cyan-intense}{HTML}{258F8F}
    \definecolor{ansi-white}{HTML}{C5C1B4}
    \definecolor{ansi-white-intense}{HTML}{A1A6B2}

    % commands and environments needed by pandoc snippets
    % extracted from the output of `pandoc -s`
    \providecommand{\tightlist}{%
      \setlength{\itemsep}{0pt}\setlength{\parskip}{0pt}}
    \DefineVerbatimEnvironment{Highlighting}{Verbatim}{commandchars=\\\{\}}
    % Add ',fontsize=\small' for more characters per line
    \newenvironment{Shaded}{}{}
    \newcommand{\KeywordTok}[1]{\textcolor[rgb]{0.00,0.44,0.13}{\textbf{{#1}}}}
    \newcommand{\DataTypeTok}[1]{\textcolor[rgb]{0.56,0.13,0.00}{{#1}}}
    \newcommand{\DecValTok}[1]{\textcolor[rgb]{0.25,0.63,0.44}{{#1}}}
    \newcommand{\BaseNTok}[1]{\textcolor[rgb]{0.25,0.63,0.44}{{#1}}}
    \newcommand{\FloatTok}[1]{\textcolor[rgb]{0.25,0.63,0.44}{{#1}}}
    \newcommand{\CharTok}[1]{\textcolor[rgb]{0.25,0.44,0.63}{{#1}}}
    \newcommand{\StringTok}[1]{\textcolor[rgb]{0.25,0.44,0.63}{{#1}}}
    \newcommand{\CommentTok}[1]{\textcolor[rgb]{0.38,0.63,0.69}{\textit{{#1}}}}
    \newcommand{\OtherTok}[1]{\textcolor[rgb]{0.00,0.44,0.13}{{#1}}}
    \newcommand{\AlertTok}[1]{\textcolor[rgb]{1.00,0.00,0.00}{\textbf{{#1}}}}
    \newcommand{\FunctionTok}[1]{\textcolor[rgb]{0.02,0.16,0.49}{{#1}}}
    \newcommand{\RegionMarkerTok}[1]{{#1}}
    \newcommand{\ErrorTok}[1]{\textcolor[rgb]{1.00,0.00,0.00}{\textbf{{#1}}}}
    \newcommand{\NormalTok}[1]{{#1}}
    
    % Additional commands for more recent versions of Pandoc
    \newcommand{\ConstantTok}[1]{\textcolor[rgb]{0.53,0.00,0.00}{{#1}}}
    \newcommand{\SpecialCharTok}[1]{\textcolor[rgb]{0.25,0.44,0.63}{{#1}}}
    \newcommand{\VerbatimStringTok}[1]{\textcolor[rgb]{0.25,0.44,0.63}{{#1}}}
    \newcommand{\SpecialStringTok}[1]{\textcolor[rgb]{0.73,0.40,0.53}{{#1}}}
    \newcommand{\ImportTok}[1]{{#1}}
    \newcommand{\DocumentationTok}[1]{\textcolor[rgb]{0.73,0.13,0.13}{\textit{{#1}}}}
    \newcommand{\AnnotationTok}[1]{\textcolor[rgb]{0.38,0.63,0.69}{\textbf{\textit{{#1}}}}}
    \newcommand{\CommentVarTok}[1]{\textcolor[rgb]{0.38,0.63,0.69}{\textbf{\textit{{#1}}}}}
    \newcommand{\VariableTok}[1]{\textcolor[rgb]{0.10,0.09,0.49}{{#1}}}
    \newcommand{\ControlFlowTok}[1]{\textcolor[rgb]{0.00,0.44,0.13}{\textbf{{#1}}}}
    \newcommand{\OperatorTok}[1]{\textcolor[rgb]{0.40,0.40,0.40}{{#1}}}
    \newcommand{\BuiltInTok}[1]{{#1}}
    \newcommand{\ExtensionTok}[1]{{#1}}
    \newcommand{\PreprocessorTok}[1]{\textcolor[rgb]{0.74,0.48,0.00}{{#1}}}
    \newcommand{\AttributeTok}[1]{\textcolor[rgb]{0.49,0.56,0.16}{{#1}}}
    \newcommand{\InformationTok}[1]{\textcolor[rgb]{0.38,0.63,0.69}{\textbf{\textit{{#1}}}}}
    \newcommand{\WarningTok}[1]{\textcolor[rgb]{0.38,0.63,0.69}{\textbf{\textit{{#1}}}}}
    
    
    % Define a nice break command that doesn't care if a line doesn't already
    % exist.
    \def\br{\hspace*{\fill} \\* }
    % Math Jax compatability definitions
    \def\gt{>}
    \def\lt{<}
    % Document parameters
        
    \title{Computación evolutiva: segunda práctica}
    
    \author{Andrés Mañas Mañas}
    

    % Pygments definitions
    
\makeatletter
\def\PY@reset{\let\PY@it=\relax \let\PY@bf=\relax%
    \let\PY@ul=\relax \let\PY@tc=\relax%
    \let\PY@bc=\relax \let\PY@ff=\relax}
\def\PY@tok#1{\csname PY@tok@#1\endcsname}
\def\PY@toks#1+{\ifx\relax#1\empty\else%
    \PY@tok{#1}\expandafter\PY@toks\fi}
\def\PY@do#1{\PY@bc{\PY@tc{\PY@ul{%
    \PY@it{\PY@bf{\PY@ff{#1}}}}}}}
\def\PY#1#2{\PY@reset\PY@toks#1+\relax+\PY@do{#2}}

\expandafter\def\csname PY@tok@gd\endcsname{\def\PY@tc##1{\textcolor[rgb]{0.63,0.00,0.00}{##1}}}
\expandafter\def\csname PY@tok@gu\endcsname{\let\PY@bf=\textbf\def\PY@tc##1{\textcolor[rgb]{0.50,0.00,0.50}{##1}}}
\expandafter\def\csname PY@tok@gt\endcsname{\def\PY@tc##1{\textcolor[rgb]{0.00,0.27,0.87}{##1}}}
\expandafter\def\csname PY@tok@gs\endcsname{\let\PY@bf=\textbf}
\expandafter\def\csname PY@tok@gr\endcsname{\def\PY@tc##1{\textcolor[rgb]{1.00,0.00,0.00}{##1}}}
\expandafter\def\csname PY@tok@cm\endcsname{\let\PY@it=\textit\def\PY@tc##1{\textcolor[rgb]{0.25,0.50,0.50}{##1}}}
\expandafter\def\csname PY@tok@vg\endcsname{\def\PY@tc##1{\textcolor[rgb]{0.10,0.09,0.49}{##1}}}
\expandafter\def\csname PY@tok@vi\endcsname{\def\PY@tc##1{\textcolor[rgb]{0.10,0.09,0.49}{##1}}}
\expandafter\def\csname PY@tok@mh\endcsname{\def\PY@tc##1{\textcolor[rgb]{0.40,0.40,0.40}{##1}}}
\expandafter\def\csname PY@tok@cs\endcsname{\let\PY@it=\textit\def\PY@tc##1{\textcolor[rgb]{0.25,0.50,0.50}{##1}}}
\expandafter\def\csname PY@tok@ge\endcsname{\let\PY@it=\textit}
\expandafter\def\csname PY@tok@vc\endcsname{\def\PY@tc##1{\textcolor[rgb]{0.10,0.09,0.49}{##1}}}
\expandafter\def\csname PY@tok@il\endcsname{\def\PY@tc##1{\textcolor[rgb]{0.40,0.40,0.40}{##1}}}
\expandafter\def\csname PY@tok@go\endcsname{\def\PY@tc##1{\textcolor[rgb]{0.53,0.53,0.53}{##1}}}
\expandafter\def\csname PY@tok@cp\endcsname{\def\PY@tc##1{\textcolor[rgb]{0.74,0.48,0.00}{##1}}}
\expandafter\def\csname PY@tok@gi\endcsname{\def\PY@tc##1{\textcolor[rgb]{0.00,0.63,0.00}{##1}}}
\expandafter\def\csname PY@tok@gh\endcsname{\let\PY@bf=\textbf\def\PY@tc##1{\textcolor[rgb]{0.00,0.00,0.50}{##1}}}
\expandafter\def\csname PY@tok@ni\endcsname{\let\PY@bf=\textbf\def\PY@tc##1{\textcolor[rgb]{0.60,0.60,0.60}{##1}}}
\expandafter\def\csname PY@tok@nl\endcsname{\def\PY@tc##1{\textcolor[rgb]{0.63,0.63,0.00}{##1}}}
\expandafter\def\csname PY@tok@nn\endcsname{\let\PY@bf=\textbf\def\PY@tc##1{\textcolor[rgb]{0.00,0.00,1.00}{##1}}}
\expandafter\def\csname PY@tok@no\endcsname{\def\PY@tc##1{\textcolor[rgb]{0.53,0.00,0.00}{##1}}}
\expandafter\def\csname PY@tok@na\endcsname{\def\PY@tc##1{\textcolor[rgb]{0.49,0.56,0.16}{##1}}}
\expandafter\def\csname PY@tok@nb\endcsname{\def\PY@tc##1{\textcolor[rgb]{0.00,0.50,0.00}{##1}}}
\expandafter\def\csname PY@tok@nc\endcsname{\let\PY@bf=\textbf\def\PY@tc##1{\textcolor[rgb]{0.00,0.00,1.00}{##1}}}
\expandafter\def\csname PY@tok@nd\endcsname{\def\PY@tc##1{\textcolor[rgb]{0.67,0.13,1.00}{##1}}}
\expandafter\def\csname PY@tok@ne\endcsname{\let\PY@bf=\textbf\def\PY@tc##1{\textcolor[rgb]{0.82,0.25,0.23}{##1}}}
\expandafter\def\csname PY@tok@nf\endcsname{\def\PY@tc##1{\textcolor[rgb]{0.00,0.00,1.00}{##1}}}
\expandafter\def\csname PY@tok@si\endcsname{\let\PY@bf=\textbf\def\PY@tc##1{\textcolor[rgb]{0.73,0.40,0.53}{##1}}}
\expandafter\def\csname PY@tok@s2\endcsname{\def\PY@tc##1{\textcolor[rgb]{0.73,0.13,0.13}{##1}}}
\expandafter\def\csname PY@tok@nt\endcsname{\let\PY@bf=\textbf\def\PY@tc##1{\textcolor[rgb]{0.00,0.50,0.00}{##1}}}
\expandafter\def\csname PY@tok@nv\endcsname{\def\PY@tc##1{\textcolor[rgb]{0.10,0.09,0.49}{##1}}}
\expandafter\def\csname PY@tok@s1\endcsname{\def\PY@tc##1{\textcolor[rgb]{0.73,0.13,0.13}{##1}}}
\expandafter\def\csname PY@tok@ch\endcsname{\let\PY@it=\textit\def\PY@tc##1{\textcolor[rgb]{0.25,0.50,0.50}{##1}}}
\expandafter\def\csname PY@tok@m\endcsname{\def\PY@tc##1{\textcolor[rgb]{0.40,0.40,0.40}{##1}}}
\expandafter\def\csname PY@tok@gp\endcsname{\let\PY@bf=\textbf\def\PY@tc##1{\textcolor[rgb]{0.00,0.00,0.50}{##1}}}
\expandafter\def\csname PY@tok@sh\endcsname{\def\PY@tc##1{\textcolor[rgb]{0.73,0.13,0.13}{##1}}}
\expandafter\def\csname PY@tok@ow\endcsname{\let\PY@bf=\textbf\def\PY@tc##1{\textcolor[rgb]{0.67,0.13,1.00}{##1}}}
\expandafter\def\csname PY@tok@sx\endcsname{\def\PY@tc##1{\textcolor[rgb]{0.00,0.50,0.00}{##1}}}
\expandafter\def\csname PY@tok@bp\endcsname{\def\PY@tc##1{\textcolor[rgb]{0.00,0.50,0.00}{##1}}}
\expandafter\def\csname PY@tok@c1\endcsname{\let\PY@it=\textit\def\PY@tc##1{\textcolor[rgb]{0.25,0.50,0.50}{##1}}}
\expandafter\def\csname PY@tok@o\endcsname{\def\PY@tc##1{\textcolor[rgb]{0.40,0.40,0.40}{##1}}}
\expandafter\def\csname PY@tok@kc\endcsname{\let\PY@bf=\textbf\def\PY@tc##1{\textcolor[rgb]{0.00,0.50,0.00}{##1}}}
\expandafter\def\csname PY@tok@c\endcsname{\let\PY@it=\textit\def\PY@tc##1{\textcolor[rgb]{0.25,0.50,0.50}{##1}}}
\expandafter\def\csname PY@tok@mf\endcsname{\def\PY@tc##1{\textcolor[rgb]{0.40,0.40,0.40}{##1}}}
\expandafter\def\csname PY@tok@err\endcsname{\def\PY@bc##1{\setlength{\fboxsep}{0pt}\fcolorbox[rgb]{1.00,0.00,0.00}{1,1,1}{\strut ##1}}}
\expandafter\def\csname PY@tok@mb\endcsname{\def\PY@tc##1{\textcolor[rgb]{0.40,0.40,0.40}{##1}}}
\expandafter\def\csname PY@tok@ss\endcsname{\def\PY@tc##1{\textcolor[rgb]{0.10,0.09,0.49}{##1}}}
\expandafter\def\csname PY@tok@sr\endcsname{\def\PY@tc##1{\textcolor[rgb]{0.73,0.40,0.53}{##1}}}
\expandafter\def\csname PY@tok@mo\endcsname{\def\PY@tc##1{\textcolor[rgb]{0.40,0.40,0.40}{##1}}}
\expandafter\def\csname PY@tok@kd\endcsname{\let\PY@bf=\textbf\def\PY@tc##1{\textcolor[rgb]{0.00,0.50,0.00}{##1}}}
\expandafter\def\csname PY@tok@mi\endcsname{\def\PY@tc##1{\textcolor[rgb]{0.40,0.40,0.40}{##1}}}
\expandafter\def\csname PY@tok@kn\endcsname{\let\PY@bf=\textbf\def\PY@tc##1{\textcolor[rgb]{0.00,0.50,0.00}{##1}}}
\expandafter\def\csname PY@tok@cpf\endcsname{\let\PY@it=\textit\def\PY@tc##1{\textcolor[rgb]{0.25,0.50,0.50}{##1}}}
\expandafter\def\csname PY@tok@kr\endcsname{\let\PY@bf=\textbf\def\PY@tc##1{\textcolor[rgb]{0.00,0.50,0.00}{##1}}}
\expandafter\def\csname PY@tok@s\endcsname{\def\PY@tc##1{\textcolor[rgb]{0.73,0.13,0.13}{##1}}}
\expandafter\def\csname PY@tok@kp\endcsname{\def\PY@tc##1{\textcolor[rgb]{0.00,0.50,0.00}{##1}}}
\expandafter\def\csname PY@tok@w\endcsname{\def\PY@tc##1{\textcolor[rgb]{0.73,0.73,0.73}{##1}}}
\expandafter\def\csname PY@tok@kt\endcsname{\def\PY@tc##1{\textcolor[rgb]{0.69,0.00,0.25}{##1}}}
\expandafter\def\csname PY@tok@sc\endcsname{\def\PY@tc##1{\textcolor[rgb]{0.73,0.13,0.13}{##1}}}
\expandafter\def\csname PY@tok@sb\endcsname{\def\PY@tc##1{\textcolor[rgb]{0.73,0.13,0.13}{##1}}}
\expandafter\def\csname PY@tok@k\endcsname{\let\PY@bf=\textbf\def\PY@tc##1{\textcolor[rgb]{0.00,0.50,0.00}{##1}}}
\expandafter\def\csname PY@tok@se\endcsname{\let\PY@bf=\textbf\def\PY@tc##1{\textcolor[rgb]{0.73,0.40,0.13}{##1}}}
\expandafter\def\csname PY@tok@sd\endcsname{\let\PY@it=\textit\def\PY@tc##1{\textcolor[rgb]{0.73,0.13,0.13}{##1}}}

\def\PYZbs{\char`\\}
\def\PYZus{\char`\_}
\def\PYZob{\char`\{}
\def\PYZcb{\char`\}}
\def\PYZca{\char`\^}
\def\PYZam{\char`\&}
\def\PYZlt{\char`\<}
\def\PYZgt{\char`\>}
\def\PYZsh{\char`\#}
\def\PYZpc{\char`\%}
\def\PYZdl{\char`\$}
\def\PYZhy{\char`\-}
\def\PYZsq{\char`\'}
\def\PYZdq{\char`\"}
\def\PYZti{\char`\~}
% for compatibility with earlier versions
\def\PYZat{@}
\def\PYZlb{[}
\def\PYZrb{]}
\makeatother


    % Exact colors from NB
    \definecolor{incolor}{rgb}{0.0, 0.0, 0.5}
    \definecolor{outcolor}{rgb}{0.545, 0.0, 0.0}



    
    % Prevent overflowing lines due to hard-to-break entities
    \sloppy 
    % Setup hyperref package
    \hypersetup{
      breaklinks=true,  % so long urls are correctly broken across lines
      colorlinks=true,
      urlcolor=urlcolor,
      linkcolor=linkcolor,
      citecolor=citecolor,
      }
    % Slightly bigger margins than the latex defaults
    
    \geometry{verbose,tmargin=1in,bmargin=1in,lmargin=1in,rmargin=1in}
    
    

\begin{document}


\begin{titlepage}


    \maketitle

 
    \begin{abstract}
    En este documento se desarrolla y prueba una estrategia evolutiva para la búsqueda de los mínimos globales de las funciones Hiper Elipsoid Rotado y función de Rastrigin para 10 dimensiones. \newline \newline \hspace{15pt} Se representan las funciones para el caso de 2 dimensiones. \newline \newline \hspace{15pt} Se adjunta y comenta el código de la estrategia evolutiva y se describen las características de la aplicación implementada. \newline \newline \hspace{15pt} Se aplica la estrategia para buscar los mínimos de las funciones anteriores en el caso de 10 dimensiones. \newline \newline \hspace{15pt} Se describen los experimentos realizados para determinar los mejores parámetros del algoritmo para la búsqueda de la solución. \newline \newline \hspace{15pt} Se adjunta tabla de resultados y conclusiones a las que se llega al amparo de los mismos.
    \end{abstract}


\thispagestyle{empty}
\end{titlepage}

\clearpage

\setlength{\parindent}{30pt}

\headheight 40pt            
\headsep 20pt   

\lhead{Computación evolutiva: segunda práctica}
\rhead{Andrés Mañas Mañas}
\pagestyle{fancy}

\setcounter{page}{1}

    
    \section{Tabla de contenidos}\label{tabla-de-contenidos}

{1~~}Ejercicio 1: Representación gráfica de las funciones para dos
dimensiones

{2~~}Ejercicio 2: Implementación de la estrategia evolutiva de búsqueda
de mínimos globales

{2.1~~}Introducción

{2.2~~}Implementación

{2.3~~}Descripción de la estrategia

{2.4~~}Pruebas en dos dimensiones

{2.4.1~~}Hiperelipsoide rotado en 2 dimensiones

{2.4.2~~}Rastrigin en 2 dimensiones

{3~~}Ejercicio 3: Búsqueda del mínimo del hiper-elipsoide rotado de 10
dimensiones con selección mu coma lambda y paso único

{4~~}Ejercicio 4: Búsqueda del mínimo del hiper-elipsoide rotado de 10
dimensiones con selección mu coma lambda y paso n

{5~~}Ejercicio 5: Búsqueda del mínimo del hiper-elipsoide rotado de 10
dimensiones con selección mu plus lambda y paso único

{6~~}Ejercicio 6: Búsqueda del mínimo del hiper-elipsoide rotado de 10
dimensiones con selección mu plus lambda y paso n

{7~~}Métricas de evaluación de configuraciones

{8~~}Ejercicio 7: Resultados para la función hiper-elipsoide rotado

{9~~}Ejercicio 9: Análisis equivalente para la función de Rastrigin

{10~~}Ejercicio 8: Búsqueda de la mejor configuración de parámetros

{11~~}Resultados, análisis y comparación

{12~~}Conclusiones

    \section{Ejercicio 1: Representación gráfica de las funciones para dos
dimensiones}\label{ejercicio-1-representaciuxf3n-gruxe1fica-de-las-funciones-para-dos-dimensiones}

Primero vamos a definir una función que nos permite dibujar en 3
dimensiones. Utilizamos la librería
\href{http://matplotlib.org/}{matplotlib}, bastante conocida en el mundo
\href{https://www.python.org/}{python}.

    \begin{Verbatim}[commandchars=\\\{\}]
{\color{incolor}In [{\color{incolor}1}]:} \PY{o}{\PYZpc{}}\PY{k}{matplotlib} inline
        
        \PY{k+kn}{import} \PY{n+nn}{numpy} \PY{k+kn}{as} \PY{n+nn}{np}
        \PY{k+kn}{from} \PY{n+nn}{mpl\PYZus{}toolkits.mplot3d} \PY{k+kn}{import} \PY{n}{Axes3D}
        \PY{k+kn}{import} \PY{n+nn}{matplotlib.pyplot} \PY{k+kn}{as} \PY{n+nn}{plt}
        \PY{k+kn}{import} \PY{n+nn}{random}
        \PY{k+kn}{import} \PY{n+nn}{math} 
        \PY{k+kn}{from} \PY{n+nn}{matplotlib} \PY{k+kn}{import} \PY{n}{cm}
        
        
        \PY{k}{def} \PY{n+nf}{plot\PYZus{}3d}\PY{p}{(}\PY{n}{f}\PY{p}{,}\PY{n}{a}\PY{p}{,}\PY{n}{b}\PY{p}{,}\PY{n}{step}\PY{p}{,}\PY{n}{figsize}\PY{o}{=}\PY{p}{(}\PY{l+m+mi}{8}\PY{p}{,} \PY{l+m+mi}{8}\PY{p}{)}\PY{p}{)}\PY{p}{:}
            \PY{l+s+sd}{\PYZdq{}\PYZdq{}\PYZdq{}}
        \PY{l+s+sd}{    Dibuja en tres dimensiones una función }
        \PY{l+s+sd}{    de dos variables}
        
        \PY{l+s+sd}{    Args:}
        \PY{l+s+sd}{        f (function): la función}
        \PY{l+s+sd}{        a (float): extremo izquierdo del }
        \PY{l+s+sd}{            dominio para las dos variables de f}
        \PY{l+s+sd}{        b (float): extremo derecho del dominio }
        \PY{l+s+sd}{            para  las dos variables de f}
        \PY{l+s+sd}{        step (float): granularidad con la }
        \PY{l+s+sd}{            que se  muestrea f}
        \PY{l+s+sd}{        figsize (int,int): tamaño de la figura}
        \PY{l+s+sd}{    \PYZdq{}\PYZdq{}\PYZdq{}}
            \PY{n}{x} \PY{o}{=} \PY{n}{y} \PY{o}{=} \PY{n}{np}\PY{o}{.}\PY{n}{arange}\PY{p}{(}\PY{n}{a}\PY{p}{,}\PY{n}{b}\PY{p}{,}\PY{n}{step}\PY{p}{)}
            \PY{n}{X}\PY{p}{,} \PY{n}{Y} \PY{o}{=} \PY{n}{np}\PY{o}{.}\PY{n}{meshgrid}\PY{p}{(}\PY{n}{x}\PY{p}{,} \PY{n}{y}\PY{p}{)}
            \PY{n}{z} \PY{o}{=} \PY{n}{np}\PY{o}{.}\PY{n}{array}\PY{p}{(}\PY{p}{[}\PY{n}{f}\PY{p}{(}\PY{p}{[}\PY{n}{x}\PY{p}{,}\PY{n}{y}\PY{p}{]}\PY{p}{)} \PY{k}{for} \PY{n}{x}\PY{p}{,}\PY{n}{y} \PY{o+ow}{in} \PY{n+nb}{zip}\PY{p}{(}\PY{n}{np}\PY{o}{.}\PY{n}{ravel}\PY{p}{(}\PY{n}{X}\PY{p}{)}\PY{p}{,} 
                                                  \PY{n}{np}\PY{o}{.}\PY{n}{ravel}\PY{p}{(}\PY{n}{Y}\PY{p}{)}\PY{p}{)}\PY{p}{]}\PY{p}{)}
            \PY{n}{Z} \PY{o}{=} \PY{n}{z}\PY{o}{.}\PY{n}{reshape}\PY{p}{(}\PY{n}{X}\PY{o}{.}\PY{n}{shape}\PY{p}{)}
                
            \PY{n}{fig} \PY{o}{=} \PY{n}{plt}\PY{o}{.}\PY{n}{figure}\PY{p}{(}\PY{n}{figsize}\PY{o}{=}\PY{n}{figsize}\PY{p}{)}
            \PY{n}{ax} \PY{o}{=} \PY{n}{fig}\PY{o}{.}\PY{n}{add\PYZus{}subplot}\PY{p}{(}\PY{l+m+mi}{111}\PY{p}{,} \PY{n}{projection}\PY{o}{=}\PY{l+s+s1}{\PYZsq{}}\PY{l+s+s1}{3d}\PY{l+s+s1}{\PYZsq{}}\PY{p}{)}
            \PY{n}{ax}\PY{o}{.}\PY{n}{set\PYZus{}xlabel}\PY{p}{(}\PY{l+s+s1}{\PYZsq{}}\PY{l+s+s1}{X}\PY{l+s+s1}{\PYZsq{}}\PY{p}{)}
            \PY{n}{ax}\PY{o}{.}\PY{n}{set\PYZus{}ylabel}\PY{p}{(}\PY{l+s+s1}{\PYZsq{}}\PY{l+s+s1}{Y}\PY{l+s+s1}{\PYZsq{}}\PY{p}{)}
            \PY{n}{ax}\PY{o}{.}\PY{n}{set\PYZus{}zlabel}\PY{p}{(}\PY{l+s+s1}{\PYZsq{}}\PY{l+s+s1}{Z}\PY{l+s+s1}{\PYZsq{}}\PY{p}{)}
            \PY{n}{ax}\PY{o}{.}\PY{n}{plot\PYZus{}surface}\PY{p}{(}\PY{n}{X}\PY{p}{,} \PY{n}{Y}\PY{p}{,} \PY{n}{Z}\PY{p}{,} \PY{n}{cmap}\PY{o}{=}\PY{n}{cm}\PY{o}{.}\PY{n}{coolwarm\PYZus{}r}\PY{p}{)}   
\end{Verbatim}

    Defino ahora la función hiper-elipsoide rotado y la función de Rastrgin.

    \begin{Verbatim}[commandchars=\\\{\}]
{\color{incolor}In [{\color{incolor}2}]:} \PY{k}{def} \PY{n+nf}{hiper\PYZus{}elipsoide\PYZus{}rotado}\PY{p}{(}\PY{n}{coords}\PY{p}{)}\PY{p}{:}
            \PY{l+s+sd}{\PYZdq{}\PYZdq{}\PYZdq{}}
        \PY{l+s+sd}{    Devuelve el valor del hiper\PYZhy{}elipsoide }
        \PY{l+s+sd}{    rotado en el punto dado por coords.}
        
        \PY{l+s+sd}{    Args:}
        \PY{l+s+sd}{        coords: las coordenadas de un }
        \PY{l+s+sd}{            punto en un espacio real de }
        \PY{l+s+sd}{            dimensión n}
        \PY{l+s+sd}{    \PYZdq{}\PYZdq{}\PYZdq{}}
            \PY{n}{result} \PY{o}{=} \PY{l+m+mi}{0}
            \PY{k}{for} \PY{n}{i} \PY{o+ow}{in} \PY{n+nb}{range}\PY{p}{(}\PY{l+m+mi}{1}\PY{p}{,} \PY{n+nb}{len}\PY{p}{(}\PY{n}{coords}\PY{p}{)}\PY{o}{+}\PY{l+m+mi}{1}\PY{p}{)}\PY{p}{:}
                \PY{k}{for} \PY{n}{j} \PY{o+ow}{in} \PY{n+nb}{range}\PY{p}{(}\PY{l+m+mi}{1}\PY{p}{,} \PY{n}{i}\PY{o}{+}\PY{l+m+mi}{1}\PY{p}{)}\PY{p}{:}
                    \PY{n}{result} \PY{o}{+}\PY{o}{=} \PY{n}{coords}\PY{p}{[}\PY{n}{j}\PY{o}{\PYZhy{}}\PY{l+m+mi}{1}\PY{p}{]}\PY{o}{*}\PY{o}{*}\PY{l+m+mi}{2}
            \PY{k}{return} \PY{n}{result}
        
        \PY{k}{def} \PY{n+nf}{rastrigin}\PY{p}{(}\PY{n}{coords}\PY{p}{)}\PY{p}{:}
            \PY{l+s+sd}{\PYZdq{}\PYZdq{}\PYZdq{}}
        \PY{l+s+sd}{    Devuelve el valor de la función de }
        \PY{l+s+sd}{    Rastrigin en el punto dado por coords.}
        
        \PY{l+s+sd}{    Args:}
        \PY{l+s+sd}{        coords: las coordenadas de un punto }
        \PY{l+s+sd}{            en un espacio real de dimensión n}
        \PY{l+s+sd}{    \PYZdq{}\PYZdq{}\PYZdq{}}
            \PY{n}{result} \PY{o}{=} \PY{l+m+mi}{0}
            \PY{k}{for} \PY{n}{xi} \PY{o+ow}{in} \PY{n}{coords}\PY{p}{:}
                \PY{n}{result} \PY{o}{+}\PY{o}{=} \PY{p}{(}\PY{n}{xi}\PY{o}{*}\PY{o}{*}\PY{l+m+mi}{2} \PY{o}{\PYZhy{}} \PY{l+m+mi}{10}\PY{o}{*}\PY{n}{math}\PY{o}{.}\PY{n}{cos}\PY{p}{(}\PY{l+m+mi}{2}\PY{o}{*}\PY{n}{math}\PY{o}{.}\PY{n}{pi}\PY{o}{*}\PY{n}{xi}\PY{p}{)}\PY{p}{)}
            \PY{k}{return} \PY{l+m+mi}{10}\PY{o}{*}\PY{n+nb}{len}\PY{p}{(}\PY{n}{coords}\PY{p}{)} \PY{o}{+} \PY{n}{result}
\end{Verbatim}

    Podemos dibujar ahora el hiper-elipsoide rotado de 2 dimensiones:

    \begin{Verbatim}[commandchars=\\\{\}]
{\color{incolor}In [{\color{incolor}30}]:} \PY{n}{plot\PYZus{}3d}\PY{p}{(}\PY{n}{hiper\PYZus{}elipsoide\PYZus{}rotado}\PY{p}{,} \PY{o}{\PYZhy{}}\PY{l+m+mf}{65.54} \PY{p}{,} \PY{l+m+mf}{65.54} \PY{p}{,} \PY{l+m+mi}{1}\PY{p}{)}
\end{Verbatim}

    \begin{center}
    \adjustimage{max size={0.9\linewidth}{0.9\paperheight}}{p2_files/p2_6_0.png}
    \end{center}
    { \hspace*{\fill} \\}
    
    Y la función de Rastrigin:

    \begin{Verbatim}[commandchars=\\\{\}]
{\color{incolor}In [{\color{incolor}31}]:} \PY{n}{plot\PYZus{}3d}\PY{p}{(}\PY{n}{rastrigin}\PY{p}{,} \PY{o}{\PYZhy{}}\PY{l+m+mf}{5.12} \PY{p}{,} \PY{l+m+mf}{5.12} \PY{p}{,} \PY{l+m+mf}{0.02}\PY{p}{)}
\end{Verbatim}

    \begin{center}
    \adjustimage{max size={0.9\linewidth}{0.9\paperheight}}{p2_files/p2_8_0.png}
    \end{center}
    { \hspace*{\fill} \\}
    
    En el caso de la función de Rastrigin, claramente vemos que el
renderizado en 3 dimensiones no nos dá suficiente información visual
(tiene tantos máximos y mínimos que acaba uno por no saber como es la
función).

Por eso, será conveniente disponer de la siguiente función que nos
permite tener una vista cenital de las funciones representando el valor
que adoptan las mismas como un mapa de calor.

    \begin{Verbatim}[commandchars=\\\{\}]
{\color{incolor}In [{\color{incolor}3}]:} \PY{k}{def} \PY{n+nf}{plot\PYZus{}color\PYZus{}map}\PY{p}{(}\PY{n}{f}\PY{p}{,}\PY{n}{a}\PY{p}{,}\PY{n}{b}\PY{p}{,}\PY{n}{step}\PY{p}{,}\PY{n}{figsize}\PY{o}{=}\PY{p}{(}\PY{l+m+mi}{10}\PY{p}{,} \PY{l+m+mi}{6}\PY{p}{)}\PY{p}{)}\PY{p}{:}
            \PY{l+s+sd}{\PYZdq{}\PYZdq{}\PYZdq{}}
        \PY{l+s+sd}{    Dibuja en dos dimensiones la vista }
        \PY{l+s+sd}{    cenital de una función de dos variables }
        \PY{l+s+sd}{    vista como un mapa de intensidad.}
        
        \PY{l+s+sd}{    Args:}
        \PY{l+s+sd}{        f: la función}
        \PY{l+s+sd}{        a: extremo izquierdo del dominio }
        \PY{l+s+sd}{            para las dos variables de f}
        \PY{l+s+sd}{        b: extremo derecho del dominio }
        \PY{l+s+sd}{            para las dos variables de f}
        \PY{l+s+sd}{        step: granularidad con la que se}
        \PY{l+s+sd}{            muestrea f}
        \PY{l+s+sd}{        figsize: tamaño de la figura}
        \PY{l+s+sd}{    \PYZdq{}\PYZdq{}\PYZdq{}}
            \PY{n}{x} \PY{o}{=} \PY{n}{y} \PY{o}{=} \PY{n}{np}\PY{o}{.}\PY{n}{arange}\PY{p}{(}\PY{n}{a}\PY{p}{,}\PY{n}{b}\PY{p}{,}\PY{n}{step}\PY{p}{)}
            \PY{n}{X}\PY{p}{,} \PY{n}{Y} \PY{o}{=} \PY{n}{np}\PY{o}{.}\PY{n}{meshgrid}\PY{p}{(}\PY{n}{x}\PY{p}{,} \PY{n}{y}\PY{p}{)}
            \PY{n}{z} \PY{o}{=} \PY{n}{np}\PY{o}{.}\PY{n}{array}\PY{p}{(}\PY{p}{[}\PY{n}{f}\PY{p}{(}\PY{p}{[}\PY{n}{x}\PY{p}{,}\PY{n}{y}\PY{p}{]}\PY{p}{)} \PY{k}{for} \PY{n}{x}\PY{p}{,}\PY{n}{y} 
                          \PY{o+ow}{in} \PY{n+nb}{zip}\PY{p}{(}\PY{n}{np}\PY{o}{.}\PY{n}{ravel}\PY{p}{(}\PY{n}{X}\PY{p}{)}\PY{p}{,} \PY{n}{np}\PY{o}{.}\PY{n}{ravel}\PY{p}{(}\PY{n}{Y}\PY{p}{)}\PY{p}{)}\PY{p}{]}\PY{p}{)}
            \PY{n}{Z} \PY{o}{=} \PY{n}{z}\PY{o}{.}\PY{n}{reshape}\PY{p}{(}\PY{n}{X}\PY{o}{.}\PY{n}{shape}\PY{p}{)}
        
            \PY{n}{fig} \PY{o}{=} \PY{n}{plt}\PY{o}{.}\PY{n}{figure}\PY{p}{(}\PY{n}{figsize}\PY{o}{=}\PY{n}{figsize}\PY{p}{)}
            \PY{n}{ax} \PY{o}{=} \PY{n}{fig}\PY{o}{.}\PY{n}{add\PYZus{}subplot}\PY{p}{(}\PY{l+m+mi}{111}\PY{p}{)}
            \PY{n}{ax}\PY{o}{.}\PY{n}{set\PYZus{}xlabel}\PY{p}{(}\PY{l+s+s1}{\PYZsq{}}\PY{l+s+s1}{X}\PY{l+s+s1}{\PYZsq{}}\PY{p}{)}
            \PY{n}{ax}\PY{o}{.}\PY{n}{set\PYZus{}ylabel}\PY{p}{(}\PY{l+s+s1}{\PYZsq{}}\PY{l+s+s1}{Y}\PY{l+s+s1}{\PYZsq{}}\PY{p}{)}
            \PY{n}{plt}\PY{o}{.}\PY{n}{imshow}\PY{p}{(}\PY{n}{Z}\PY{p}{,}\PY{n}{extent}\PY{o}{=}\PY{p}{[}\PY{n}{a}\PY{p}{,}\PY{n}{b}\PY{p}{,}\PY{n}{a}\PY{p}{,}\PY{n}{b}\PY{p}{]}\PY{p}{)}
            \PY{n}{plt}\PY{o}{.}\PY{n}{colorbar}\PY{p}{(}\PY{n}{orientation}\PY{o}{=}\PY{l+s+s1}{\PYZsq{}}\PY{l+s+s1}{vertical}\PY{l+s+s1}{\PYZsq{}}\PY{p}{)}
\end{Verbatim}

    Con esta función podemos ver que efectivamente el hiper elipsoide rotado
se corresponde con una superficie con un único mínimo en el origen:

    \begin{Verbatim}[commandchars=\\\{\}]
{\color{incolor}In [{\color{incolor}33}]:} \PY{n}{plot\PYZus{}color\PYZus{}map}\PY{p}{(}\PY{n}{hiper\PYZus{}elipsoide\PYZus{}rotado}\PY{p}{,} \PY{o}{\PYZhy{}}\PY{l+m+mf}{65.54} \PY{p}{,} \PY{l+m+mf}{65.54} \PY{p}{,} \PY{l+m+mi}{1}\PY{p}{)}
\end{Verbatim}

    \begin{center}
    \adjustimage{max size={0.9\linewidth}{0.9\paperheight}}{p2_files/p2_12_0.png}
    \end{center}
    { \hspace*{\fill} \\}
    
    Del mismo modo, podemos tener una vista más intuitiva del aspecto de la
función de Rastrigin para 2 dimensiones. Observamos que en el intervalo
en el que se dibuja tiene muchos mínimos locales, cuyo valor se hace más
pequeño a medida que se aproximan al origen de coordenadas donde se
encuentra su mínimo global.

    \begin{Verbatim}[commandchars=\\\{\}]
{\color{incolor}In [{\color{incolor}34}]:} \PY{n}{plot\PYZus{}color\PYZus{}map}\PY{p}{(}\PY{n}{rastrigin}\PY{p}{,} \PY{o}{\PYZhy{}}\PY{l+m+mf}{5.12} \PY{p}{,} \PY{l+m+mf}{5.12} \PY{p}{,} \PY{l+m+mf}{0.02}\PY{p}{)}
\end{Verbatim}

    \begin{center}
    \adjustimage{max size={0.9\linewidth}{0.9\paperheight}}{p2_files/p2_14_0.png}
    \end{center}
    { \hspace*{\fill} \\}
    
    \section{Ejercicio 2: Implementación de la estrategia evolutiva de
búsqueda de mínimos
globales}\label{ejercicio-2-implementaciuxf3n-de-la-estrategia-evolutiva-de-buxfasqueda-de-muxednimos-globales}

Siguiendo las indicaciones de la práctica, para la búsqueda de los
mínimos globales de las funciones hiper elipsoide rotado y función de
Rastrigin, se ha inplementado una estrategia evolutiva.

** Este tipo de estratedias están principalmente indicadas en problemas
de optimización numérica (como es el caso de los problemas que queremos
resolver). Son rápidas y suelen funcionar bien en problemas en los que
se pueden representar las soluciones con números reales. **

A lo largo de la evolución, en este tipo de estrategias, no sólo va
evolucionando la solución sino también los parámetros que controlan la
evolución de la misma.

\subsection{Introducción}\label{introducciuxf3n}

Mi implemetación es lo suficientemente genérica como para resolver
problemas de cualquier dimensión n.

En el caso de mi implementación, represento los individuos como
diccionarios con las propiedades: - coords (coordenadas del individuo) -
sigmas (parámetros sigma para el operador de mutación).

Dado que en la práctica no se piden mutaciones correlacionadas, no
agrego en la representación los parámetros alfa.

Además, dado que en la práctica se piden experimentos de paso único y de
n pasos determino que la propiedad sigmas de un individuo puede: - tener
un único elemento numérico, en cuyo caso el comportamiento del algoritmo
será de paso único - o bien tener un array con el mismo número de
elementos que coords, en cuyo caso el comportamiento del algoritmo será
de n pasos

Será responsabilidad de la implementación de mis algoritmos interpretar,
según la representación de los individuos, si la computación debe
realizarse desde el enfoque de paso único (misma sigma controla la
variabilidad de todas las coordenadas) o de n pasos (un sigma
independiente para el control de la variabilidad de cada coordenada).

Por ejemplo, para el caso de 3 dimensiones, un ejemplo de individuo
perteneciente a una ejecución configurada con paso único sería:

\begin{Shaded}
\begin{Highlighting}[]
\NormalTok{individual_1 }\OperatorTok{=} \NormalTok{\{}\StringTok{'coords'}\NormalTok{:[}\DecValTok{1}\NormalTok{,}\DecValTok{2}\NormalTok{,}\DecValTok{3}\NormalTok{],}\StringTok{'sigmas'}\NormalTok{:[}\DecValTok{1}\NormalTok{]\}}
\end{Highlighting}
\end{Shaded}

Y para el mismo caso de 3 dimensiones, un ejemplo de individuo
perteneciente a una ejecución configurada con n pasos sería:

\begin{Shaded}
\begin{Highlighting}[]
\NormalTok{individual_2 }\OperatorTok{=} \NormalTok{\{}\StringTok{'coords'}\NormalTok{:[}\DecValTok{1}\NormalTok{,}\DecValTok{2}\NormalTok{,}\DecValTok{3}\NormalTok{],}\StringTok{'sigmas'}\NormalTok{:[}\DecValTok{2}\NormalTok{,}\DecValTok{3}\NormalTok{,}\DecValTok{1}\NormalTok{]\}}
\end{Highlighting}
\end{Shaded}

\subsection{Implementación}\label{implementaciuxf3n}

    \begin{Verbatim}[commandchars=\\\{\}]
{\color{incolor}In [{\color{incolor}4}]:} \PY{k+kn}{from} \PY{n+nn}{collections} \PY{k+kn}{import} \PY{n}{Iterable} 
        \PY{k+kn}{import} \PY{n+nn}{math}
        \PY{k+kn}{import} \PY{n+nn}{inspect}
        
        \PY{k}{def} \PY{n+nf}{rand\PYZus{}individual}\PY{p}{(}\PY{n}{domain}\PY{p}{,}\PY{n}{dimension}\PY{p}{,}\PY{n}{step}\PY{o}{=}\PY{l+s+s1}{\PYZsq{}}\PY{l+s+s1}{n}\PY{l+s+s1}{\PYZsq{}}\PY{p}{,}\PY{n}{sigma}\PY{o}{=}\PY{l+m+mi}{1}\PY{p}{)}\PY{p}{:}    
            \PY{l+s+sd}{\PYZdq{}\PYZdq{}\PYZdq{}}
        \PY{l+s+sd}{    Genera un inviduo aleatoriamente.}
        \PY{l+s+sd}{    Para conocer la representación utilizada para}
        \PY{l+s+sd}{    los individuos, léanse los explicaciones }
        \PY{l+s+sd}{    anteriores.}
        
        \PY{l+s+sd}{    Args:}
        \PY{l+s+sd}{        domain: intervalo real de definición de la }
        \PY{l+s+sd}{            función cuyos óptimos buscamos}
        \PY{l+s+sd}{        dimension: dimensión del experimento}
        \PY{l+s+sd}{        step: \PYZsq{}n\PYZsq{} si el individuo va a participar }
        \PY{l+s+sd}{            en una estrategia de n pasos o \PYZsq{}one\PYZsq{} }
        \PY{l+s+sd}{            si la estrategia es de paso único}
        \PY{l+s+sd}{        sigma: valor que adoptará/n el/los sigma/s }
        \PY{l+s+sd}{            del individuo dependiendo de que la }
        \PY{l+s+sd}{            estrategia sea de paso único o de n pasos}
        \PY{l+s+sd}{    \PYZdq{}\PYZdq{}\PYZdq{}}
            \PY{n}{coords}\PY{o}{=}\PY{n}{np}\PY{o}{.}\PY{n}{random}\PY{o}{.}\PY{n}{uniform}\PY{p}{(}\PY{n}{domain}\PY{p}{[}\PY{l+m+mi}{0}\PY{p}{]}\PY{p}{,}\PY{n}{domain}\PY{p}{[}\PY{l+m+mi}{1}\PY{p}{]}\PY{p}{,}\PY{n}{dimension}\PY{p}{)}
            \PY{k}{if} \PY{n}{step}\PY{o}{==}\PY{l+s+s1}{\PYZsq{}}\PY{l+s+s1}{one}\PY{l+s+s1}{\PYZsq{}}\PY{p}{:}
                \PY{n}{sigmas}\PY{o}{=}\PY{p}{[}\PY{n}{sigma}\PY{p}{]}
            \PY{k}{elif} \PY{n}{step}\PY{o}{==}\PY{l+s+s1}{\PYZsq{}}\PY{l+s+s1}{n}\PY{l+s+s1}{\PYZsq{}}\PY{p}{:}
                \PY{n}{sigmas}\PY{o}{=}\PY{p}{[}\PY{n}{sigma}\PY{p}{]}\PY{o}{*}\PY{n}{dimension}
            \PY{k}{else}\PY{p}{:}
                \PY{k}{raise} \PY{n+ne}{ValueError}\PY{p}{(}\PY{l+s+s1}{\PYZsq{}}\PY{l+s+s1}{Not a valid step}\PY{l+s+s1}{\PYZsq{}}\PY{p}{)}
            \PY{k}{return} \PY{p}{\PYZob{}}\PY{l+s+s1}{\PYZsq{}}\PY{l+s+s1}{coords}\PY{l+s+s1}{\PYZsq{}}\PY{p}{:}\PY{n}{coords}\PY{p}{,}\PY{l+s+s1}{\PYZsq{}}\PY{l+s+s1}{sigmas}\PY{l+s+s1}{\PYZsq{}}\PY{p}{:}\PY{n}{sigmas}\PY{p}{\PYZcb{}}
        
        
        \PY{k}{def} \PY{n+nf}{one\PYZus{}step\PYZus{}mutation}\PY{p}{(}\PY{n}{individual}\PY{p}{,}\PY{n}{min\PYZus{}sigma}\PY{p}{,}\PY{n}{tau}\PY{p}{,}\PY{n}{tau\PYZus{}prima}\PY{o}{=}\PY{n+nb+bp}{None}\PY{p}{)}\PY{p}{:}
            \PY{l+s+sd}{\PYZdq{}\PYZdq{}\PYZdq{}}
        \PY{l+s+sd}{    Muta un individuo en experimentos }
        \PY{l+s+sd}{    de paso único.}
        
        \PY{l+s+sd}{    Args:}
        \PY{l+s+sd}{        individual: el individuo a mutar}
        \PY{l+s+sd}{        min\PYZus{}sigma: el valor mínimo de sigma }
        \PY{l+s+sd}{            permitido en el individuo mutado}
        \PY{l+s+sd}{        generation: generacion en curso}
        \PY{l+s+sd}{        tau: intensidad de mutación}
        \PY{l+s+sd}{        tau\PYZus{}prima: se ignora}
        \PY{l+s+sd}{    \PYZdq{}\PYZdq{}\PYZdq{}}
            \PY{n}{coords}\PY{p}{,}\PY{p}{[}\PY{n}{sigma}\PY{p}{]}\PY{o}{=}\PY{n}{individual}\PY{p}{[}\PY{l+s+s1}{\PYZsq{}}\PY{l+s+s1}{coords}\PY{l+s+s1}{\PYZsq{}}\PY{p}{]}\PY{p}{,}\PY{n}{individual}\PY{p}{[}\PY{l+s+s1}{\PYZsq{}}\PY{l+s+s1}{sigmas}\PY{l+s+s1}{\PYZsq{}}\PY{p}{]}
            \PY{n}{sigma\PYZus{}mut}\PY{o}{=}\PY{n}{sigma}\PY{o}{*}\PY{n}{math}\PY{o}{.}\PY{n}{exp}\PY{p}{(}\PY{n}{tau}\PY{o}{*}\PY{n}{np}\PY{o}{.}\PY{n}{random}\PY{o}{.}\PY{n}{normal}\PY{p}{(}\PY{p}{)}\PY{p}{)}
            \PY{n}{sigma\PYZus{}mut}\PY{o}{=}\PY{n}{sigma\PYZus{}mut} \PY{k}{if} \PY{n}{min\PYZus{}sigma}\PY{o}{\PYZlt{}}\PY{n}{sigma\PYZus{}mut} \PY{k}{else} \PY{n}{sigma}
            \PY{n}{coords\PYZus{}mut}\PY{o}{=}\PY{p}{[}\PY{n}{x}\PY{o}{+}\PY{n}{sigma\PYZus{}mut}\PY{o}{*}\PY{n}{np}\PY{o}{.}\PY{n}{random}\PY{o}{.}\PY{n}{normal}\PY{p}{(}\PY{p}{)} \PY{k}{for} \PY{n}{x} \PY{o+ow}{in} \PY{n}{coords}\PY{p}{]}
            \PY{k}{return} \PY{p}{\PYZob{}}\PY{l+s+s1}{\PYZsq{}}\PY{l+s+s1}{coords}\PY{l+s+s1}{\PYZsq{}}\PY{p}{:}\PY{n}{coords\PYZus{}mut}\PY{p}{,}\PY{l+s+s1}{\PYZsq{}}\PY{l+s+s1}{sigmas}\PY{l+s+s1}{\PYZsq{}}\PY{p}{:}\PY{p}{[}\PY{n}{sigma\PYZus{}mut}\PY{p}{]}\PY{p}{\PYZcb{}}
        
        
        \PY{k}{def} \PY{n+nf}{n\PYZus{}step\PYZus{}mutation}\PY{p}{(}\PY{n}{individual}\PY{p}{,}\PY{n}{min\PYZus{}sigma}\PY{p}{,}\PY{n}{tau}\PY{p}{,}\PY{n}{tau\PYZus{}prima}\PY{p}{)}\PY{p}{:}
            \PY{l+s+sd}{\PYZdq{}\PYZdq{}\PYZdq{}}
        \PY{l+s+sd}{    Muta un individuo en experimentos de n pasos.}
        
        \PY{l+s+sd}{    Args:}
        \PY{l+s+sd}{        individual: el individuo a mutar}
        \PY{l+s+sd}{        min\PYZus{}sigma: el valor mínimo de sigma }
        \PY{l+s+sd}{            permitido en los sigmas del }
        \PY{l+s+sd}{            individuo mutado}
        \PY{l+s+sd}{        generation: generacion en curso}
        \PY{l+s+sd}{        tau: intensidad de mutación común}
        \PY{l+s+sd}{        tau\PYZus{}prima: intensidad de mutación }
        \PY{l+s+sd}{            por componente}
        \PY{l+s+sd}{    \PYZdq{}\PYZdq{}\PYZdq{}}
            \PY{n}{coords}\PY{p}{,}\PY{n}{sigmas}\PY{o}{=}\PY{n}{individual}\PY{p}{[}\PY{l+s+s1}{\PYZsq{}}\PY{l+s+s1}{coords}\PY{l+s+s1}{\PYZsq{}}\PY{p}{]}\PY{p}{,}\PY{n}{individual}\PY{p}{[}\PY{l+s+s1}{\PYZsq{}}\PY{l+s+s1}{sigmas}\PY{l+s+s1}{\PYZsq{}}\PY{p}{]}
            \PY{n}{common\PYZus{}norm}\PY{o}{=}\PY{n}{np}\PY{o}{.}\PY{n}{random}\PY{o}{.}\PY{n}{normal}\PY{p}{(}\PY{p}{)}
            \PY{n}{sigmas\PYZus{}mut}\PY{o}{=}\PY{p}{[}\PY{p}{]}
            \PY{n}{coords\PYZus{}mut}\PY{o}{=}\PY{p}{[}\PY{p}{]}
            \PY{k}{for} \PY{n}{sigma}\PY{p}{,}\PY{n}{coord} \PY{o+ow}{in} \PY{n+nb}{zip}\PY{p}{(}\PY{n}{sigmas}\PY{p}{,}\PY{n}{coords}\PY{p}{)}\PY{p}{:}
                \PY{n}{sigma\PYZus{}mut}\PY{o}{=}\PY{n}{sigma}\PY{o}{*}\PY{n}{math}\PY{o}{.}\PY{n}{exp}\PY{p}{(}\PY{n}{tau\PYZus{}prima}\PY{o}{*}\PY{n}{common\PYZus{}norm} \PY{o}{+} 
                                         \PY{n}{tau}\PY{o}{*}\PY{n}{np}\PY{o}{.}\PY{n}{random}\PY{o}{.}\PY{n}{normal}\PY{p}{(}\PY{p}{)}\PY{p}{)}
                \PY{n}{sigma\PYZus{}mut}\PY{o}{=}\PY{n}{sigma\PYZus{}mut} \PY{k}{if} \PY{n}{min\PYZus{}sigma}\PY{o}{\PYZlt{}}\PY{n}{sigma\PYZus{}mut} \PY{k}{else} \PY{n}{sigma}
                \PY{n}{coord\PYZus{}mut}\PY{o}{=}\PY{n}{coord} \PY{o}{+} \PY{n}{sigma\PYZus{}mut}\PY{o}{*}\PY{n}{np}\PY{o}{.}\PY{n}{random}\PY{o}{.}\PY{n}{normal}\PY{p}{(}\PY{p}{)}
                \PY{n}{sigmas\PYZus{}mut} \PY{o}{+}\PY{o}{=} \PY{p}{[}\PY{n}{sigma\PYZus{}mut}\PY{p}{]}
                \PY{n}{coords\PYZus{}mut} \PY{o}{+}\PY{o}{=} \PY{p}{[}\PY{n}{coord\PYZus{}mut}\PY{p}{]}
            \PY{k}{return} \PY{p}{\PYZob{}}\PY{l+s+s1}{\PYZsq{}}\PY{l+s+s1}{coords}\PY{l+s+s1}{\PYZsq{}}\PY{p}{:}\PY{n}{coords\PYZus{}mut}\PY{p}{,}\PY{l+s+s1}{\PYZsq{}}\PY{l+s+s1}{sigmas}\PY{l+s+s1}{\PYZsq{}}\PY{p}{:}\PY{n}{sigmas\PYZus{}mut}\PY{p}{\PYZcb{}}
        
        
        \PY{k}{def} \PY{n+nf}{discrete\PYZus{}recombination}\PY{p}{(}\PY{n}{ind\PYZus{}1}\PY{p}{,}\PY{n}{ind\PYZus{}2}\PY{p}{)}\PY{p}{:}
            \PY{l+s+sd}{\PYZdq{}\PYZdq{}\PYZdq{}}
        \PY{l+s+sd}{    Recombina dos individuos por el método }
        \PY{l+s+sd}{    discreto (tomando una a una la coordenada }
        \PY{l+s+sd}{    y el sigma de uno u otro individuo, }
        \PY{l+s+sd}{    según el azar).}
        \PY{l+s+sd}{    Este método acepta tanto individuos que }
        \PY{l+s+sd}{    participan en estrategias de paso único }
        \PY{l+s+sd}{    como de n pasos.}
        
        \PY{l+s+sd}{    Args:}
        \PY{l+s+sd}{        ind\PYZus{}1: primer individuo}
        \PY{l+s+sd}{        ind\PYZus{}2: segundo individuo}
        \PY{l+s+sd}{        }
        \PY{l+s+sd}{    Returns:}
        \PY{l+s+sd}{        la recombinación de ambos individuos}
        \PY{l+s+sd}{    \PYZdq{}\PYZdq{}\PYZdq{}}
            \PY{n}{sigmas\PYZus{}comb}\PY{o}{=}\PY{p}{[}\PY{p}{]}
            \PY{n}{coords\PYZus{}comb}\PY{o}{=}\PY{p}{[}\PY{p}{]}
            \PY{n}{both}\PY{o}{=}\PY{p}{[}\PY{n}{ind\PYZus{}1}\PY{p}{,}\PY{n}{ind\PYZus{}2}\PY{p}{]}
            \PY{n}{one\PYZus{}step}\PY{o}{=} \PY{l+m+mi}{1}\PY{o}{==}\PY{n+nb}{len}\PY{p}{(}\PY{n}{ind\PYZus{}1}\PY{p}{[}\PY{l+s+s1}{\PYZsq{}}\PY{l+s+s1}{sigmas}\PY{l+s+s1}{\PYZsq{}}\PY{p}{]}\PY{p}{)}
            \PY{k}{for} \PY{n}{i} \PY{o+ow}{in} \PY{n+nb}{range}\PY{p}{(}\PY{n+nb}{len}\PY{p}{(}\PY{n}{ind\PYZus{}1}\PY{p}{[}\PY{l+s+s1}{\PYZsq{}}\PY{l+s+s1}{coords}\PY{l+s+s1}{\PYZsq{}}\PY{p}{]}\PY{p}{)}\PY{p}{)}\PY{p}{:}
                \PY{n}{j}\PY{o}{=}\PY{n}{np}\PY{o}{.}\PY{n}{random}\PY{o}{.}\PY{n}{choice}\PY{p}{(}\PY{p}{[}\PY{l+m+mi}{0}\PY{p}{,}\PY{l+m+mi}{1}\PY{p}{]}\PY{p}{)}
                \PY{n}{coords\PYZus{}comb}\PY{o}{+}\PY{o}{=}\PY{p}{[}\PY{n}{both}\PY{p}{[}\PY{n}{j}\PY{p}{]}\PY{p}{[}\PY{l+s+s1}{\PYZsq{}}\PY{l+s+s1}{coords}\PY{l+s+s1}{\PYZsq{}}\PY{p}{]}\PY{p}{[}\PY{n}{i}\PY{p}{]}\PY{p}{]}
                \PY{k}{if} \PY{n}{one\PYZus{}step}\PY{p}{:}
                    \PY{n}{sigmas\PYZus{}comb}\PY{o}{=}\PY{p}{[}\PY{n}{both}\PY{p}{[}\PY{n}{j}\PY{p}{]}\PY{p}{[}\PY{l+s+s1}{\PYZsq{}}\PY{l+s+s1}{sigmas}\PY{l+s+s1}{\PYZsq{}}\PY{p}{]}\PY{p}{[}\PY{l+m+mi}{0}\PY{p}{]}\PY{p}{]}
                \PY{k}{else}\PY{p}{:}
                    \PY{n}{sigmas\PYZus{}comb}\PY{o}{+}\PY{o}{=}\PY{p}{[}\PY{n}{both}\PY{p}{[}\PY{n}{j}\PY{p}{]}\PY{p}{[}\PY{l+s+s1}{\PYZsq{}}\PY{l+s+s1}{sigmas}\PY{l+s+s1}{\PYZsq{}}\PY{p}{]}\PY{p}{[}\PY{n}{i}\PY{p}{]}\PY{p}{]}
            \PY{k}{return} \PY{p}{\PYZob{}}\PY{l+s+s1}{\PYZsq{}}\PY{l+s+s1}{coords}\PY{l+s+s1}{\PYZsq{}}\PY{p}{:}\PY{n}{coords\PYZus{}comb}\PY{p}{,}\PY{l+s+s1}{\PYZsq{}}\PY{l+s+s1}{sigmas}\PY{l+s+s1}{\PYZsq{}}\PY{p}{:}\PY{n}{sigmas\PYZus{}comb}\PY{p}{\PYZcb{}}
        
        
        \PY{k}{def} \PY{n+nf}{intermediate\PYZus{}recombination}\PY{p}{(}\PY{n}{ind\PYZus{}1}\PY{p}{,}\PY{n}{ind\PYZus{}2}\PY{p}{)}\PY{p}{:}
            \PY{l+s+sd}{\PYZdq{}\PYZdq{}\PYZdq{}}
        \PY{l+s+sd}{    Recombina dos individuos por el método }
        \PY{l+s+sd}{    intermedio (tomando una a una la }
        \PY{l+s+sd}{    coordenada y el sigma mediosde ambos }
        \PY{l+s+sd}{    individuos).}
        \PY{l+s+sd}{    Este método acepta tanto individuos que }
        \PY{l+s+sd}{    participan en estrategias de paso único }
        \PY{l+s+sd}{    como de n pasos.}
        
        \PY{l+s+sd}{    Args:}
        \PY{l+s+sd}{        ind\PYZus{}1: primer individuo}
        \PY{l+s+sd}{        ind\PYZus{}2: segundo individuo}
        \PY{l+s+sd}{    \PYZdq{}\PYZdq{}\PYZdq{}}
            \PY{k}{return} \PY{p}{\PYZob{}}\PY{l+s+s1}{\PYZsq{}}\PY{l+s+s1}{coords}\PY{l+s+s1}{\PYZsq{}}\PY{p}{:}\PY{p}{[}\PY{p}{(}\PY{n}{a}\PY{o}{+}\PY{n}{b}\PY{p}{)}\PY{o}{/}\PY{l+m+mi}{2} \PY{k}{for} \PY{n}{a}\PY{p}{,}\PY{n}{b} \PY{o+ow}{in} 
                              \PY{n+nb}{zip}\PY{p}{(}\PY{n}{ind\PYZus{}1}\PY{p}{[}\PY{l+s+s1}{\PYZsq{}}\PY{l+s+s1}{coords}\PY{l+s+s1}{\PYZsq{}}\PY{p}{]}\PY{p}{,}\PY{n}{ind\PYZus{}2}\PY{p}{[}\PY{l+s+s1}{\PYZsq{}}\PY{l+s+s1}{coords}\PY{l+s+s1}{\PYZsq{}}\PY{p}{]}\PY{p}{)}\PY{p}{]}\PY{p}{,}
                    \PY{l+s+s1}{\PYZsq{}}\PY{l+s+s1}{sigmas}\PY{l+s+s1}{\PYZsq{}}\PY{p}{:}\PY{p}{[}\PY{p}{(}\PY{n}{a}\PY{o}{+}\PY{n}{b}\PY{p}{)}\PY{o}{/}\PY{l+m+mi}{2} \PY{k}{for} \PY{n}{a}\PY{p}{,}\PY{n}{b} \PY{o+ow}{in} 
                              \PY{n+nb}{zip}\PY{p}{(}\PY{n}{ind\PYZus{}1}\PY{p}{[}\PY{l+s+s1}{\PYZsq{}}\PY{l+s+s1}{sigmas}\PY{l+s+s1}{\PYZsq{}}\PY{p}{]}\PY{p}{,}\PY{n}{ind\PYZus{}2}\PY{p}{[}\PY{l+s+s1}{\PYZsq{}}\PY{l+s+s1}{sigmas}\PY{l+s+s1}{\PYZsq{}}\PY{p}{]}\PY{p}{)}\PY{p}{]}\PY{p}{\PYZcb{}}
        
        
        
        \PY{k}{def} \PY{n+nf}{mu\PYZus{}comma\PYZus{}lambda}\PY{p}{(}\PY{n}{mus}\PY{p}{,}\PY{n}{lambdas}\PY{p}{,}\PY{n}{f}\PY{p}{,}\PY{n}{mu}\PY{p}{)}\PY{p}{:}
            \PY{l+s+sd}{\PYZdq{}\PYZdq{}\PYZdq{}}
        \PY{l+s+sd}{    Selección de tipo mu comma lambda}
        \PY{l+s+sd}{    (selecciona los mejores individuos del }
        \PY{l+s+sd}{    conjunto de lambas/hijos).}
        
        \PY{l+s+sd}{    Args:}
        \PY{l+s+sd}{        mus: los padres de la generación }
        \PY{l+s+sd}{            en curso}
        \PY{l+s+sd}{        lambdas: los hijos de la generación }
        \PY{l+s+sd}{            en curso }
        \PY{l+s+sd}{        f: la función de evaluación de la }
        \PY{l+s+sd}{            calidad de los individuos }
        \PY{l+s+sd}{            (función fitness)}
        \PY{l+s+sd}{        mu: número de individuos a seleccionar}
        \PY{l+s+sd}{    \PYZdq{}\PYZdq{}\PYZdq{}}
            \PY{k}{return} \PY{n+nb}{sorted}\PY{p}{(}\PY{n}{lambdas}\PY{p}{,} \PY{n}{key}\PY{o}{=}\PY{k}{lambda} \PY{n}{x}\PY{p}{:}\PY{n}{f}\PY{p}{(}\PY{n}{x}\PY{p}{[}\PY{l+s+s1}{\PYZsq{}}\PY{l+s+s1}{coords}\PY{l+s+s1}{\PYZsq{}}\PY{p}{]}\PY{p}{)}\PY{p}{)}\PY{p}{[}\PY{l+m+mi}{0}\PY{p}{:}\PY{n}{mu}\PY{p}{]}
        
        
        \PY{k}{def} \PY{n+nf}{mu\PYZus{}plus\PYZus{}lambda}\PY{p}{(}\PY{n}{mus}\PY{p}{,}\PY{n}{lambdas}\PY{p}{,}\PY{n}{f}\PY{p}{,}\PY{n}{mu}\PY{p}{)}\PY{p}{:}
            \PY{l+s+sd}{\PYZdq{}\PYZdq{}\PYZdq{}}
        \PY{l+s+sd}{    Selección de tipo mu plus lambda}
        \PY{l+s+sd}{    (selecciona los mejores individuos del }
        \PY{l+s+sd}{    conjunto de mus/padres y lambas/hijos).}
        
        \PY{l+s+sd}{    Args:}
        \PY{l+s+sd}{        mus: los padres de la generación en curso}
        \PY{l+s+sd}{        lambdas: los hijos de la generación en curso }
        \PY{l+s+sd}{        f: la función de evaluación de la calidad}
        \PY{l+s+sd}{            de los individuos (función fitness)}
        \PY{l+s+sd}{        mu: número de individuos a seleccionar}
        \PY{l+s+sd}{    \PYZdq{}\PYZdq{}\PYZdq{}}
            \PY{k}{return} \PY{n+nb}{sorted}\PY{p}{(}\PY{n}{mus}\PY{o}{+}\PY{n}{lambdas}\PY{p}{,} 
                          \PY{n}{key}\PY{o}{=}\PY{k}{lambda} \PY{n}{x}\PY{p}{:}\PY{n}{f}\PY{p}{(}\PY{n}{x}\PY{p}{[}\PY{l+s+s1}{\PYZsq{}}\PY{l+s+s1}{coords}\PY{l+s+s1}{\PYZsq{}}\PY{p}{]}\PY{p}{)}\PY{p}{)}\PY{p}{[}\PY{l+m+mi}{0}\PY{p}{:}\PY{n}{mu}\PY{p}{]}
        
        
        \PY{k}{def} \PY{n+nf}{termination\PYZus{}cond\PYZus{}met}\PY{p}{(}\PY{n}{best}\PY{p}{,}\PY{n}{generation}\PY{p}{,}\PY{n}{max\PYZus{}generations}\PY{p}{,}
                                 \PY{n}{termination\PYZus{}delta}\PY{p}{,}\PY{n}{f}\PY{p}{)}\PY{p}{:}
            \PY{l+s+sd}{\PYZdq{}\PYZdq{}\PYZdq{}}
        \PY{l+s+sd}{    Determina si se dan las condiciones para que }
        \PY{l+s+sd}{    la evolución acabe por}
        \PY{l+s+sd}{     \PYZhy{} haber evolucionado durante demasiadas }
        \PY{l+s+sd}{         generaciones}
        \PY{l+s+sd}{     \PYZhy{} haber alcanzado el mejor individuo de la }
        \PY{l+s+sd}{         generación en curso un fitness }
        \PY{l+s+sd}{         suficientemente bueno}
        
        \PY{l+s+sd}{    Args:}
        \PY{l+s+sd}{        best: array con el histórico de los }
        \PY{l+s+sd}{            mejores individuos de cada generación}
        \PY{l+s+sd}{        generation: generación en curso}
        \PY{l+s+sd}{        max\PYZus{}generations: máximo número de }
        \PY{l+s+sd}{            generaciones permitido en el experimento}
        \PY{l+s+sd}{        termination\PYZus{}delta: error mínimo del mejor }
        \PY{l+s+sd}{            individuo. Si el fitness del mejor }
        \PY{l+s+sd}{            individuo (cercanía al mínimo global }
        \PY{l+s+sd}{            que sabemos que es 0) es inferior }
        \PY{l+s+sd}{            a este valor, el experimento se da }
        \PY{l+s+sd}{            por concluido.}
        \PY{l+s+sd}{        f: la función de evaluación de la calidad}
        \PY{l+s+sd}{            de los individuos (función fitness)}
        \PY{l+s+sd}{    \PYZdq{}\PYZdq{}\PYZdq{}}    
            \PY{k}{return} \PY{n}{max\PYZus{}generations}\PY{o}{\PYZlt{}}\PY{o}{=}\PY{n}{generation} \PYZbs{}
                   \PY{o+ow}{or} \PY{l+m+mi}{0}\PY{o}{\PYZlt{}}\PY{n+nb}{len}\PY{p}{(}\PY{n}{best}\PY{p}{)} \PYZbs{}
                      \PY{o+ow}{and} \PY{n+nb}{abs}\PY{p}{(}\PY{n}{f}\PY{p}{(}\PY{n}{best}\PY{p}{[}\PY{o}{\PYZhy{}}\PY{l+m+mi}{1}\PY{p}{]}\PY{p}{[}\PY{l+s+s1}{\PYZsq{}}\PY{l+s+s1}{coords}\PY{l+s+s1}{\PYZsq{}}\PY{p}{]}\PY{p}{)}\PY{p}{)}\PY{o}{\PYZlt{}}\PY{n}{termination\PYZus{}delta}
        
        
        \PY{k}{def} \PY{n+nf}{live}\PY{p}{(}\PY{n}{domain}\PY{p}{,} \PY{n}{dimension}\PY{p}{,} \PY{n}{f}\PY{p}{,} 
                 \PY{n}{step}\PY{o}{=}\PY{l+s+s1}{\PYZsq{}}\PY{l+s+s1}{n}\PY{l+s+s1}{\PYZsq{}}\PY{p}{,} 
                 \PY{n}{selection}\PY{o}{=}\PY{l+s+s1}{\PYZsq{}}\PY{l+s+s1}{mu\PYZus{}comma\PYZus{}lambda}\PY{l+s+s1}{\PYZsq{}}\PY{p}{,} 
                 \PY{n}{recombination}\PY{o}{=}\PY{l+s+s1}{\PYZsq{}}\PY{l+s+s1}{intermediate}\PY{l+s+s1}{\PYZsq{}}\PY{p}{,}
                 \PY{n}{mu}\PY{o}{=}\PY{l+m+mi}{30}\PY{p}{,} 
                 \PY{n}{lamb}\PY{o}{=}\PY{l+m+mi}{200}\PY{p}{,}
                 \PY{n}{sigma}\PY{o}{=}\PY{l+m+mi}{1}\PY{p}{,} 
                 \PY{n}{min\PYZus{}sigma}\PY{o}{=}\PY{o}{.}\PY{l+m+mi}{0}\PY{p}{,} 
                 \PY{n}{max\PYZus{}generations}\PY{o}{=}\PY{l+m+mi}{1000}\PY{p}{,}
                 \PY{n}{termination\PYZus{}delta}\PY{o}{=}\PY{o}{.}\PY{l+m+mi}{0}\PY{p}{,}
                 \PY{n}{tau\PYZus{}factor}\PY{o}{=}\PY{l+m+mi}{1}\PY{p}{,}
                 \PY{n}{tau\PYZus{}prima\PYZus{}factor}\PY{o}{=}\PY{l+m+mi}{1}\PY{p}{,}
                 \PY{n}{mutate\PYZus{}sigmas\PYZus{}every}\PY{o}{=}\PY{l+m+mi}{1}\PY{p}{)}\PY{p}{:}    
            \PY{l+s+sd}{\PYZdq{}\PYZdq{}\PYZdq{}}
        \PY{l+s+sd}{    Lleva a cabo la estrategia evolutiva de }
        \PY{l+s+sd}{    búsqueda de mínimos globales de las funciones }
        \PY{l+s+sd}{    hiper elipsoide rotado o función de Rastrigin, }
        \PY{l+s+sd}{    tal como se ha descrito  en el presente }
        \PY{l+s+sd}{    documento.}
        
        \PY{l+s+sd}{    Args:}
        \PY{l+s+sd}{        domain: dominio de definición de la }
        \PY{l+s+sd}{            función cuyos mínimos globales }
        \PY{l+s+sd}{            se buscan}
        \PY{l+s+sd}{        dimension: demensión del espacio en el }
        \PY{l+s+sd}{            que se define la función f}
        \PY{l+s+sd}{        f: función cuyos mínimos globales }
        \PY{l+s+sd}{            buscamos con la estrategia evolutiva}
        \PY{l+s+sd}{        step: determina si la estrategia será }
        \PY{l+s+sd}{            de paso único \PYZsq{}one\PYZsq{} o de n pasos \PYZsq{}n\PYZsq{}. }
        \PY{l+s+sd}{            El valor por defecto es \PYZsq{}n\PYZsq{}.}
        \PY{l+s+sd}{        selection: determina si la selección }
        \PY{l+s+sd}{            de hijos  se hará con el método }
        \PY{l+s+sd}{            mu comma lambda \PYZsq{}mu\PYZus{}comma\PYZus{}lambda\PYZsq{} }
        \PY{l+s+sd}{            o mu plus lambda  \PYZsq{}mu\PYZus{}plus\PYZus{}lambda\PYZsq{}.}
        \PY{l+s+sd}{            Valor por defecto \PYZsq{}mu\PYZus{}comma\PYZus{}lambda\PYZsq{}.}
        \PY{l+s+sd}{        recombination: determina si la recombinación }
        \PY{l+s+sd}{            de individuos se hará por el método }
        \PY{l+s+sd}{            discreto \PYZsq{}discrete\PYZsq{} o intermedio}
        \PY{l+s+sd}{            \PYZsq{}intermediate\PYZsq{}.}
        \PY{l+s+sd}{            Valor por defecto \PYZsq{}intermediate\PYZsq{}}
        \PY{l+s+sd}{        mu: número de padres en cada iteración. }
        \PY{l+s+sd}{            El valor por defecto es de 30.}
        \PY{l+s+sd}{        lamb: número de hijos en cada interación.}
        \PY{l+s+sd}{            El valor por defecto es de 200.}
        \PY{l+s+sd}{        sigma: valor por defecto que se aplicará }
        \PY{l+s+sd}{            en la  inicialización de los sigmas }
        \PY{l+s+sd}{            de todo individuo.}
        \PY{l+s+sd}{            Valor por defecto 1.}
        \PY{l+s+sd}{        min\PYZus{}sigma: valor mínimo permitido de los }
        \PY{l+s+sd}{            sigmas de un individuo.}
        \PY{l+s+sd}{            Valor por defecto 0.0001.}
        \PY{l+s+sd}{        max\PYZus{}generations: número de generaciones }
        \PY{l+s+sd}{            máximo permitido en el experimento. }
        \PY{l+s+sd}{            Valor por defecto 1000:}
        \PY{l+s+sd}{        termination\PYZus{}delta: error mínimo del mejor }
        \PY{l+s+sd}{            individuo. }
        \PY{l+s+sd}{            Si el fitness del mejor individuo }
        \PY{l+s+sd}{            (cercanía al mínimo global que sabemos}
        \PY{l+s+sd}{            que es 0) es inferior a este valor, }
        \PY{l+s+sd}{            el experimento se da por concluido.}
        \PY{l+s+sd}{            Valor por defecto .0      }
        \PY{l+s+sd}{        tau\PYZus{}factor: factor de multiplicación del}
        \PY{l+s+sd}{            valor por defecto de tau, 1.}
        \PY{l+s+sd}{        tau\PYZus{}prima\PYZus{}factor: factor de multiplicación }
        \PY{l+s+sd}{            del valor por defecto de tau prima, 1.}
        \PY{l+s+sd}{            }
        \PY{l+s+sd}{    Returns:}
        \PY{l+s+sd}{        \PYZhy{} el mejor individuo encontrado tras la evolución }
        \PY{l+s+sd}{        \PYZhy{} y un array con el histórico del mejor inviduo }
        \PY{l+s+sd}{            en cada generación.}
        \PY{l+s+sd}{    \PYZdq{}\PYZdq{}\PYZdq{}}    
            
        \PY{c+c1}{\PYZsh{}     print [locals()[arg] for arg in inspect.getargspec(live).args[3:]]}
            
            \PY{k}{if} \PY{n}{step}\PY{o}{==}\PY{l+s+s1}{\PYZsq{}}\PY{l+s+s1}{n}\PY{l+s+s1}{\PYZsq{}}\PY{p}{:}
                \PY{n}{mut\PYZus{}f}\PY{o}{=}\PY{n}{n\PYZus{}step\PYZus{}mutation}
                \PY{n}{tau}\PY{o}{=}\PY{n}{tau\PYZus{}factor}\PY{o}{/}\PY{n}{math}\PY{o}{.}\PY{n}{sqrt}\PY{p}{(}\PY{l+m+mi}{2}\PY{o}{*}\PY{n}{dimension}\PY{p}{)}
                \PY{n}{tau\PYZus{}prima}\PY{o}{=}\PY{n}{tau\PYZus{}prima\PYZus{}factor}\PY{o}{/}\PY{n}{math}\PY{o}{.}\PY{n}{sqrt}\PY{p}{(}\PY{l+m+mi}{2}\PY{o}{*}\PY{n}{math}\PY{o}{.}\PY{n}{sqrt}\PY{p}{(}\PY{n}{dimension}\PY{p}{)}\PY{p}{)}
            \PY{k}{elif} \PY{n}{step}\PY{o}{==}\PY{l+s+s1}{\PYZsq{}}\PY{l+s+s1}{one}\PY{l+s+s1}{\PYZsq{}}\PY{p}{:} 
                \PY{n}{mut\PYZus{}f}\PY{o}{=}\PY{n}{one\PYZus{}step\PYZus{}mutation}
                \PY{n}{tau}\PY{o}{=}\PY{n}{tau\PYZus{}factor}\PY{o}{/}\PY{n}{math}\PY{o}{.}\PY{n}{sqrt}\PY{p}{(}\PY{n}{dimension}\PY{p}{)}
                \PY{n}{tau\PYZus{}prima}\PY{o}{=}\PY{n+nb+bp}{None}
            
            \PY{k}{if} \PY{n}{recombination}\PY{o}{==}\PY{l+s+s1}{\PYZsq{}}\PY{l+s+s1}{intermediate}\PY{l+s+s1}{\PYZsq{}}\PY{p}{:}
                \PY{n}{rec\PYZus{}f}\PY{o}{=}\PY{n}{intermediate\PYZus{}recombination}
            \PY{k}{elif} \PY{n}{recombination}\PY{o}{==}\PY{l+s+s1}{\PYZsq{}}\PY{l+s+s1}{discrete}\PY{l+s+s1}{\PYZsq{}}\PY{p}{:}
                \PY{n}{rec\PYZus{}f}\PY{o}{=}\PY{n}{discrete\PYZus{}recombination}
                
            \PY{k}{if} \PY{n}{selection}\PY{o}{==}\PY{l+s+s1}{\PYZsq{}}\PY{l+s+s1}{mu\PYZus{}comma\PYZus{}lambda}\PY{l+s+s1}{\PYZsq{}}\PY{p}{:}
                \PY{n}{sel\PYZus{}f}\PY{o}{=}\PY{n}{mu\PYZus{}comma\PYZus{}lambda}
            \PY{k}{elif} \PY{n}{selection}\PY{o}{==}\PY{l+s+s1}{\PYZsq{}}\PY{l+s+s1}{mu\PYZus{}plus\PYZus{}lambda}\PY{l+s+s1}{\PYZsq{}}\PY{p}{:}
                \PY{n}{sel\PYZus{}f}\PY{o}{=}\PY{n}{mu\PYZus{}plus\PYZus{}lambda}
                
            \PY{n}{t}\PY{p}{,}\PY{n}{best}\PY{o}{=}\PY{l+m+mi}{0}\PY{p}{,}\PY{p}{[}\PY{p}{]}
            \PY{n}{mus}\PY{o}{=}\PY{p}{[}\PY{n}{rand\PYZus{}individual}\PY{p}{(}\PY{n}{domain}\PY{p}{,}\PY{n}{dimension}\PY{p}{,}\PY{n}{step}\PY{p}{,}\PY{n}{sigma}\PY{p}{)} 
                 \PY{k}{for} \PY{n}{i} \PY{o+ow}{in} \PY{n+nb}{range}\PY{p}{(}\PY{n}{mu}\PY{p}{)}\PY{p}{]}    
            \PY{k}{while} \PY{o+ow}{not} \PY{n}{termination\PYZus{}cond\PYZus{}met}\PY{p}{(}\PY{n}{best}\PY{p}{,}\PY{n}{t}\PY{p}{,}\PY{n}{max\PYZus{}generations}\PY{p}{,}
                                           \PY{n}{termination\PYZus{}delta}\PY{p}{,}\PY{n}{f}\PY{p}{)}\PY{p}{:}
                \PY{n}{recombinations}\PY{o}{=}\PY{p}{[}\PY{n}{rec\PYZus{}f}\PY{p}{(}\PY{o}{*}\PY{n}{random}\PY{o}{.}\PY{n}{sample}\PY{p}{(}\PY{n}{mus}\PY{p}{,}\PY{l+m+mi}{2}\PY{p}{)}\PY{p}{)} 
                                \PY{k}{for} \PY{n}{i} \PY{o+ow}{in} \PY{n+nb}{range}\PY{p}{(}\PY{n}{lamb}\PY{p}{)}\PY{p}{]}
                \PY{n}{mutations}\PY{o}{=}\PY{p}{[}\PY{n}{mut\PYZus{}f}\PY{p}{(}\PY{n}{ind}\PY{p}{,}\PY{n}{min\PYZus{}sigma}\PY{p}{,}\PY{n}{tau}\PY{p}{,}\PY{n}{tau\PYZus{}prima}\PY{p}{)} 
                           \PY{k}{for} \PY{n}{ind} \PY{o+ow}{in} \PY{n}{recombinations}\PY{p}{]}
                \PY{n}{mus}\PY{o}{=}\PY{n}{sel\PYZus{}f}\PY{p}{(}\PY{n}{mus}\PY{p}{,}\PY{n}{mutations}\PY{p}{,}\PY{n}{f}\PY{p}{,}\PY{n}{mu}\PY{p}{)}
                \PY{n}{best}\PY{o}{+}\PY{o}{=}\PY{p}{[}\PY{n}{mus}\PY{p}{[}\PY{l+m+mi}{0}\PY{p}{]}\PY{p}{]}
                \PY{n}{t}\PY{o}{+}\PY{o}{=}\PY{l+m+mi}{1}
            \PY{k}{return} \PY{n}{best}\PY{p}{[}\PY{o}{\PYZhy{}}\PY{l+m+mi}{1}\PY{p}{]}\PY{p}{,}\PY{n}{best}
\end{Verbatim}

    Defino además una función que nos permite lanzar experimentos en
paralelo para reducir el tiempo de espera en completarse un conjunto de
experimentos.

    \begin{Verbatim}[commandchars=\\\{\}]
{\color{incolor}In [{\color{incolor}10}]:} \PY{k+kn}{from} \PY{n+nn}{functools} \PY{k+kn}{import} \PY{n}{partial}
         \PY{k+kn}{from} \PY{n+nn}{multiprocessing} \PY{k+kn}{import} \PY{n}{Pool}
         
         \PY{k}{def} \PY{n+nf}{p\PYZus{}exps}\PY{p}{(}\PY{n}{name}\PY{p}{,}\PY{n}{f}\PY{p}{,}\PY{n}{n}\PY{o}{=}\PY{l+m+mi}{20}\PY{p}{)}\PY{p}{:}
             \PY{l+s+sd}{\PYZdq{}\PYZdq{}\PYZdq{}}
         \PY{l+s+sd}{        Ejecuta en paralelo una batería }
         \PY{l+s+sd}{        de experimentos.}
         \PY{l+s+sd}{    \PYZdq{}\PYZdq{}\PYZdq{}}
             \PY{n}{p} \PY{o}{=} \PY{n}{Pool}\PY{p}{(}\PY{p}{)}
             \PY{n}{result} \PY{o}{=} \PY{p}{\PYZob{}}\PY{n}{name}\PY{p}{:}\PY{n}{p}\PY{o}{.}\PY{n}{map}\PY{p}{(}\PY{n}{f}\PY{p}{,}\PY{n+nb}{range}\PY{p}{(}\PY{n}{n}\PY{p}{)}\PY{p}{)}\PY{p}{\PYZcb{}}
             \PY{n}{p}\PY{o}{.}\PY{n}{close}\PY{p}{(}\PY{p}{)}    
             \PY{n}{p}\PY{o}{.}\PY{n}{terminate}\PY{p}{(}\PY{p}{)}
             \PY{k}{return} \PY{n}{result}
\end{Verbatim}

    \subsection{Descripción de la
estrategia}\label{descripciuxf3n-de-la-estrategia}

Observando la función live, que es la que coordina la evolución, se
pueden apreciar los aspectos que siguen.

Dado un experimento, es obligatorio indicar el número de dimensiones que
se van a utilizar, la función f (hiperelipsoide rotada o rastrigin) y el
dominio de definición de la función f.

Todo experimento puede realizarse de paso único o de n pasos, con
recombinación intermedia o discreta y con selección de descendencia por
\((\mu,\lambda)\) o \((\mu + \lambda)\).

Los valores de mu, lamb, sigma, min\_sigma, max\_generations o
termination\_delta se dan por defecto, aunque pueden ser configurados en
cada ejecución del experimento.

La estrategia en sí es bastante simple, un individuo se representa por
unas coordenadas en el espacion n-dimensional y por uno (`one' step) o
tantos sigmas como coordenadas tenga el individuo (`n' step), que
gobierna la intensidad de la mutación tanto de las coordenadas como de
los propios sigmas.

En esta implementación, por hacerla más sencilla de entender y de
mantener, se toman los valores de tau por defecto que se indican en el
documento base de referencia. No se permite parametrizar tales valores
sino que todos los experimentos tendrán los mismos en función del número
de dimensiones.

Con estas explicaciones, el algoritmo simplemente hace esto: - genera
una población inicial al azar en el intervalo de definición de la
función - mientras el fitness del mejor individuo no sea suficientemente
bueno o no se haya alcanzado el máximo de generaciones permitido -
generar lamd individuos recombinando la población actual - mutar los
individuos lamb anteriores - establecer la nueva población como los
mejores individuos seleccionados de entre los individuos mutados y la
población actual (dependiendo de la función de selección) - guardar el
mejor de los resultantes en el paso anterior en el histórico de mejores
individuos por generación - repetir hasta que no se tenga un individuo
con suficiente fitness o no se hayan agotado el máximo de generaciones
permitidas - devolvel el mejor individuo y el histórico de mejores
individuos por generación

\subsection{Pruebas en dos
dimensiones}\label{pruebas-en-dos-dimensiones}

Antes de pasar a hacer los experimentos en 10 dimensiones que se piden
en la práctica es mejor ver como se comporta el algoritmo en el caso de
2 dimensiones.

La ventaja de hacerlo en 2 dimensiones es que podemos disponer de un
gráfico qeu muestre visualmente ``la traza'' de la evolución.

A tal efecto, la siguiente función nos será muy útil junto a la función
plot\_color\_map que se definión anteriormente.

    \begin{Verbatim}[commandchars=\\\{\}]
{\color{incolor}In [{\color{incolor}6}]:} \PY{k}{def} \PY{n+nf}{add\PYZus{}scatter}\PY{p}{(}\PY{n}{individuals}\PY{p}{)}\PY{p}{:}
            \PY{l+s+sd}{\PYZdq{}\PYZdq{}\PYZdq{}}
        \PY{l+s+sd}{    Dibuja un scatter con la secuencia de }
        \PY{l+s+sd}{    coordenadas que se corresponden con una }
        \PY{l+s+sd}{    secuencia de individuos según los }
        \PY{l+s+sd}{    represento en la presente práctica}
        \PY{l+s+sd}{    (se explica más arriba).}
        \PY{l+s+sd}{    Permite representar el camino que sigue la }
        \PY{l+s+sd}{    evolución de la mejor solución que el }
        \PY{l+s+sd}{    algoritmo encuentra en cada generación.}
        
        \PY{l+s+sd}{    Args:}
        \PY{l+s+sd}{        individuals: array de individuos}
        \PY{l+s+sd}{    \PYZdq{}\PYZdq{}\PYZdq{}}    
            \PY{n}{x}\PY{p}{,}\PY{n}{y}\PY{o}{=}\PY{p}{[}\PY{p}{]}\PY{p}{,}\PY{p}{[}\PY{p}{]}
            \PY{k}{for} \PY{n}{ind} \PY{o+ow}{in} \PY{n}{individuals}\PY{p}{:}
                \PY{n}{x}\PY{o}{+}\PY{o}{=}\PY{p}{[}\PY{n}{ind}\PY{p}{[}\PY{l+s+s1}{\PYZsq{}}\PY{l+s+s1}{coords}\PY{l+s+s1}{\PYZsq{}}\PY{p}{]}\PY{p}{[}\PY{l+m+mi}{0}\PY{p}{]}\PY{p}{]}
                \PY{n}{y}\PY{o}{+}\PY{o}{=}\PY{p}{[}\PY{n}{ind}\PY{p}{[}\PY{l+s+s1}{\PYZsq{}}\PY{l+s+s1}{coords}\PY{l+s+s1}{\PYZsq{}}\PY{p}{]}\PY{p}{[}\PY{l+m+mi}{1}\PY{p}{]}\PY{p}{]}
            \PY{n}{colors}\PY{o}{=}\PY{n}{cm}\PY{o}{.}\PY{n}{rainbow}\PY{p}{(}\PY{n}{np}\PY{o}{.}\PY{n}{linspace}\PY{p}{(}\PY{l+m+mi}{0}\PY{p}{,} \PY{l+m+mi}{1}\PY{p}{,} \PY{n+nb}{len}\PY{p}{(}\PY{n}{x}\PY{p}{)}\PY{p}{)}\PY{p}{)}
            \PY{n}{plt}\PY{o}{.}\PY{n}{scatter}\PY{p}{(}\PY{n}{x}\PY{p}{,} \PY{n}{y}\PY{p}{,} \PY{n}{c}\PY{o}{=}\PY{n}{colors}\PY{p}{,} \PY{n}{s}\PY{o}{=}\PY{l+m+mi}{100}\PY{p}{)}
\end{Verbatim}

    Igualmente, nos será uy útil una función que dibuje como de próximo está
un individuo al óptimo global buscado, que sabemos que en ambos casos es
0. Algo que nos sirva para representar la evolución del fitness del
mejor individuo.

\textbf{Nótese que la siguiente función hace una transformación
logarítmica de los valores por f de cada individuo, para evitar que los
gráficos salgan aberrados y por lo tanto que no sirvan para
interpretarse con provecho.}

    \begin{Verbatim}[commandchars=\\\{\}]
{\color{incolor}In [{\color{incolor}7}]:} \PY{k}{def} \PY{n+nf}{plot\PYZus{}inv\PYZus{}log\PYZus{}fitness}\PY{p}{(}\PY{n}{f}\PY{p}{,}\PY{n}{individuals}\PY{p}{)}\PY{p}{:}
            \PY{l+s+sd}{\PYZdq{}\PYZdq{}\PYZdq{}}
        \PY{l+s+sd}{    Dibuja el valor dado por f en una población }
        \PY{l+s+sd}{    de individuos en escala logarítmica. }
        \PY{l+s+sd}{    Puede interpretarse con el inverso}
        \PY{l+s+sd}{    del fitness de los individuos.}
        \PY{l+s+sd}{    }
        \PY{l+s+sd}{    Se utiliza escala logarítmica porque si no}
        \PY{l+s+sd}{    los gráficos salen un poco aberrados.}
        
        \PY{l+s+sd}{    Args:}
        \PY{l+s+sd}{        f: una función n variada}
        \PY{l+s+sd}{        individuals: array de individuos con }
        \PY{l+s+sd}{            n coordenadas cada uno.}
        \PY{l+s+sd}{    \PYZdq{}\PYZdq{}\PYZdq{}}    
            \PY{n}{plt}\PY{o}{.}\PY{n}{plot}\PY{p}{(}\PY{p}{[}\PY{n}{math}\PY{o}{.}\PY{n}{log}\PY{p}{(}\PY{n}{f}\PY{p}{(}\PY{n}{ind}\PY{p}{[}\PY{l+s+s1}{\PYZsq{}}\PY{l+s+s1}{coords}\PY{l+s+s1}{\PYZsq{}}\PY{p}{]}\PY{p}{)}\PY{o}{+}\PY{l+m+mi}{1}\PY{p}{)} 
                      \PY{k}{for} \PY{n}{ind} \PY{o+ow}{in} \PY{n}{individuals}\PY{p}{]}\PY{p}{)}
\end{Verbatim}

    \subsubsection{Hiperelipsoide rotado en 2
dimensiones}\label{hiperelipsoide-rotado-en-2-dimensiones}

De este modo, para el hiper elipsoide rotado de 2 dimensiones, con todos
los valores por defecto, podemos llevar a cabo una búsqueda del mínimo
global:

    \begin{Verbatim}[commandchars=\\\{\}]
{\color{incolor}In [{\color{incolor}38}]:} \PY{n}{best}\PY{p}{,}\PY{n+nb}{all}\PY{o}{=}\PY{n}{live}\PY{p}{(}\PY{p}{[}\PY{o}{\PYZhy{}}\PY{l+m+mf}{65.54} \PY{p}{,} \PY{l+m+mf}{65.54}\PY{p}{]}\PY{p}{,}\PY{l+m+mi}{2}\PY{p}{,}\PY{n}{hiper\PYZus{}elipsoide\PYZus{}rotado}\PY{p}{)}
\end{Verbatim}

    Y ver gráficamente como el mejor individuo de cada generación se va
aproximando al origen de coordenadas. Nótese que el aprendizaje es tan
rápido que los individuos muy rápidamente se ubican muy próximos al
origen de coordenadas. Por eso, reduzco a {[}-5,5{]} el intervalo en el
que dibujo los gráficos.

    \begin{Verbatim}[commandchars=\\\{\}]
{\color{incolor}In [{\color{incolor}39}]:} \PY{n}{plot\PYZus{}color\PYZus{}map}\PY{p}{(}\PY{n}{hiper\PYZus{}elipsoide\PYZus{}rotado}\PY{p}{,}\PY{o}{\PYZhy{}}\PY{l+m+mi}{5}\PY{p}{,}\PY{l+m+mi}{5}\PY{p}{,}\PY{l+m+mf}{0.1}\PY{p}{)}
         \PY{n}{add\PYZus{}scatter}\PY{p}{(}\PY{n+nb}{all}\PY{p}{)}
\end{Verbatim}

    \begin{center}
    \adjustimage{max size={0.9\linewidth}{0.9\paperheight}}{p2_files/p2_26_0.png}
    \end{center}
    { \hspace*{\fill} \\}
    
    Podemos ver la evolución del fitness del mejor individuo (en escala
logarítmica e inversa):

    \begin{Verbatim}[commandchars=\\\{\}]
{\color{incolor}In [{\color{incolor}42}]:} \PY{n}{plot\PYZus{}inv\PYZus{}log\PYZus{}fitness}\PY{p}{(}\PY{n}{hiper\PYZus{}elipsoide\PYZus{}rotado}\PY{p}{,}\PY{n+nb}{all}\PY{p}{[}\PY{p}{:}\PY{l+m+mi}{10}\PY{p}{]}\PY{p}{)}
\end{Verbatim}

    \begin{center}
    \adjustimage{max size={0.9\linewidth}{0.9\paperheight}}{p2_files/p2_28_0.png}
    \end{center}
    { \hspace*{\fill} \\}
    
    Observamos que encuentra el mínimo muy rapido, en 5 generaciones.

    \subsubsection{Rastrigin en 2
dimensiones}\label{rastrigin-en-2-dimensiones}

Equivalenteme, para el caso de la función de Rastrigin, en dos
dimensiones, tenemos:

    \begin{Verbatim}[commandchars=\\\{\}]
{\color{incolor}In [{\color{incolor}44}]:} \PY{n}{best}\PY{p}{,}\PY{n+nb}{all}\PY{o}{=}\PY{n}{live}\PY{p}{(}\PY{p}{[}\PY{o}{\PYZhy{}}\PY{l+m+mf}{5.12} \PY{p}{,} \PY{l+m+mf}{5.12}\PY{p}{]}\PY{p}{,}\PY{l+m+mi}{2}\PY{p}{,}\PY{n}{rastrigin}\PY{p}{)}
         
         \PY{c+c1}{\PYZsh{} Reducimos el intervalo en el que dibujamos}
         \PY{c+c1}{\PYZsh{} para mayor claridad del dibujo}
         \PY{n}{plot\PYZus{}color\PYZus{}map}\PY{p}{(}\PY{n}{rastrigin}\PY{p}{,}\PY{o}{\PYZhy{}}\PY{l+m+mi}{2}\PY{p}{,}\PY{l+m+mi}{2}\PY{p}{,}\PY{l+m+mf}{0.1}\PY{p}{)}
         \PY{n}{add\PYZus{}scatter}\PY{p}{(}\PY{n+nb}{all}\PY{p}{)}
         \PY{n}{plt}\PY{o}{.}\PY{n}{show}\PY{p}{(}\PY{p}{)}
         \PY{n}{plot\PYZus{}inv\PYZus{}log\PYZus{}fitness}\PY{p}{(}\PY{n}{rastrigin}\PY{p}{,}\PY{n+nb}{all}\PY{p}{[}\PY{p}{:}\PY{l+m+mi}{50}\PY{p}{]}\PY{p}{)}
\end{Verbatim}

    \begin{center}
    \adjustimage{max size={0.9\linewidth}{0.9\paperheight}}{p2_files/p2_31_0.png}
    \end{center}
    { \hspace*{\fill} \\}
    
    \begin{center}
    \adjustimage{max size={0.9\linewidth}{0.9\paperheight}}{p2_files/p2_31_1.png}
    \end{center}
    { \hspace*{\fill} \\}
    
    Nótese que en el caso de dos dimensiones, el mínimo de la función de
Rasttrigin también se encuentra muy rápido.

Y son esto, más o menos, podemos hacernos una idea del herramentaje que
tenemos a nuestra disposición para continuar con la práctica.

    \section{Ejercicio 3: Búsqueda del mínimo del hiper-elipsoide rotado de
10 dimensiones con selección mu coma lambda y paso
único}\label{ejercicio-3-buxfasqueda-del-muxednimo-del-hiper-elipsoide-rotado-de-10-dimensiones-con-selecciuxf3n-mu-coma-lambda-y-paso-uxfanico}

Realizamos 20 ejecuciones y guardamos los resultados en la variable
\textbf{hiper\_one\_comma}.

Analizaremos estos resultados posteriormente.

    \begin{Verbatim}[commandchars=\\\{\}]
{\color{incolor}In [{\color{incolor}9}]:} \PY{k}{def} \PY{n+nf}{partial}\PY{p}{(}\PY{n}{i}\PY{p}{)}\PY{p}{:}
            \PY{k}{return} \PY{n}{live}\PY{p}{(}\PY{p}{[}\PY{o}{\PYZhy{}}\PY{l+m+mf}{65.54} \PY{p}{,} \PY{l+m+mf}{65.54}\PY{p}{]}\PY{p}{,}\PY{l+m+mi}{10}\PY{p}{,}
                        \PY{n}{hiper\PYZus{}elipsoide\PYZus{}rotado}\PY{p}{,}
                        \PY{n}{step}\PY{o}{=}\PY{l+s+s1}{\PYZsq{}}\PY{l+s+s1}{one}\PY{l+s+s1}{\PYZsq{}}\PY{p}{,}  
                        \PY{n}{selection}\PY{o}{=}\PY{l+s+s1}{\PYZsq{}}\PY{l+s+s1}{mu\PYZus{}comma\PYZus{}lambda}\PY{l+s+s1}{\PYZsq{}}\PY{p}{)}\PY{p}{[}\PY{l+m+mi}{1}\PY{p}{]}
        \PY{n}{hiper\PYZus{}one\PYZus{}comma} \PY{o}{=}  \PY{n}{p\PYZus{}exps}\PY{p}{(}\PY{l+s+s1}{\PYZsq{}}\PY{l+s+s1}{Hiper one comma}\PY{l+s+s1}{\PYZsq{}}\PY{p}{,}\PY{n}{partial}\PY{p}{)}
\end{Verbatim}

    \section{Ejercicio 4: Búsqueda del mínimo del hiper-elipsoide rotado de
10 dimensiones con selección mu coma lambda y paso
n}\label{ejercicio-4-buxfasqueda-del-muxednimo-del-hiper-elipsoide-rotado-de-10-dimensiones-con-selecciuxf3n-mu-coma-lambda-y-paso-n}

Realizamos 20 ejecuciones y guardamos los resultados en la variable
\textbf{hiper\_n\_comma}.

Analizaremos estos resultados posteriormente.

    \begin{Verbatim}[commandchars=\\\{\}]
{\color{incolor}In [{\color{incolor}102}]:} \PY{k}{def} \PY{n+nf}{partial}\PY{p}{(}\PY{n}{i}\PY{p}{)}\PY{p}{:}
              \PY{k}{return} \PY{n}{live}\PY{p}{(}\PY{p}{[}\PY{o}{\PYZhy{}}\PY{l+m+mf}{65.54} \PY{p}{,} \PY{l+m+mf}{65.54}\PY{p}{]}\PY{p}{,}\PY{l+m+mi}{10}\PY{p}{,}
                          \PY{n}{hiper\PYZus{}elipsoide\PYZus{}rotado}\PY{p}{,}
                          \PY{n}{step}\PY{o}{=}\PY{l+s+s1}{\PYZsq{}}\PY{l+s+s1}{n}\PY{l+s+s1}{\PYZsq{}}\PY{p}{,}  
                          \PY{n}{selection}\PY{o}{=}\PY{l+s+s1}{\PYZsq{}}\PY{l+s+s1}{mu\PYZus{}comma\PYZus{}lambda}\PY{l+s+s1}{\PYZsq{}}\PY{p}{)}\PY{p}{[}\PY{l+m+mi}{1}\PY{p}{]}
          \PY{n}{hiper\PYZus{}n\PYZus{}comma} \PY{o}{=}  \PY{n}{p\PYZus{}exps}\PY{p}{(}\PY{l+s+s1}{\PYZsq{}}\PY{l+s+s1}{Hiper n   comma}\PY{l+s+s1}{\PYZsq{}}\PY{p}{,}\PY{n}{partial}\PY{p}{,}\PY{n}{n}\PY{o}{=}\PY{l+m+mi}{4}\PY{p}{)}
\end{Verbatim}

    \section{Ejercicio 5: Búsqueda del mínimo del hiper-elipsoide rotado de
10 dimensiones con selección mu plus lambda y paso
único}\label{ejercicio-5-buxfasqueda-del-muxednimo-del-hiper-elipsoide-rotado-de-10-dimensiones-con-selecciuxf3n-mu-plus-lambda-y-paso-uxfanico}

Realizamos 20 ejecuciones y guardamos los resultados en la variable
\textbf{hiper\_one\_plus}.

Analizaremos estos resultados posteriormente.

    \begin{Verbatim}[commandchars=\\\{\}]
{\color{incolor}In [{\color{incolor}103}]:} \PY{k}{def} \PY{n+nf}{partial}\PY{p}{(}\PY{n}{i}\PY{p}{)}\PY{p}{:}
              \PY{k}{return} \PY{n}{live}\PY{p}{(}\PY{p}{[}\PY{o}{\PYZhy{}}\PY{l+m+mf}{65.54} \PY{p}{,} \PY{l+m+mf}{65.54}\PY{p}{]}\PY{p}{,}\PY{l+m+mi}{10}\PY{p}{,}
                          \PY{n}{hiper\PYZus{}elipsoide\PYZus{}rotado}\PY{p}{,}
                          \PY{n}{step}\PY{o}{=}\PY{l+s+s1}{\PYZsq{}}\PY{l+s+s1}{one}\PY{l+s+s1}{\PYZsq{}}\PY{p}{,}  
                          \PY{n}{selection}\PY{o}{=}\PY{l+s+s1}{\PYZsq{}}\PY{l+s+s1}{mu\PYZus{}plus\PYZus{}lambda}\PY{l+s+s1}{\PYZsq{}}\PY{p}{)}\PY{p}{[}\PY{l+m+mi}{1}\PY{p}{]}
          \PY{n}{hiper\PYZus{}one\PYZus{}plus} \PY{o}{=}  \PY{n}{p\PYZus{}exps}\PY{p}{(}\PY{l+s+s1}{\PYZsq{}}\PY{l+s+s1}{Hiper one plus }\PY{l+s+s1}{\PYZsq{}}\PY{p}{,}\PY{n}{partial}\PY{p}{)}
\end{Verbatim}

    \section{Ejercicio 6: Búsqueda del mínimo del hiper-elipsoide rotado de
10 dimensiones con selección mu plus lambda y paso
n}\label{ejercicio-6-buxfasqueda-del-muxednimo-del-hiper-elipsoide-rotado-de-10-dimensiones-con-selecciuxf3n-mu-plus-lambda-y-paso-n}

Realizamos 20 ejecuciones y guardamos los resultados en la variable
\textbf{hiper\_n\_plus}.

Analizaremos estos resultados posteriormente.

    \begin{Verbatim}[commandchars=\\\{\}]
{\color{incolor}In [{\color{incolor}104}]:} \PY{k}{def} \PY{n+nf}{partial}\PY{p}{(}\PY{n}{i}\PY{p}{)}\PY{p}{:}
              \PY{k}{return} \PY{n}{live}\PY{p}{(}\PY{p}{[}\PY{o}{\PYZhy{}}\PY{l+m+mf}{65.54} \PY{p}{,} \PY{l+m+mf}{65.54}\PY{p}{]}\PY{p}{,}\PY{l+m+mi}{10}\PY{p}{,}
                          \PY{n}{hiper\PYZus{}elipsoide\PYZus{}rotado}\PY{p}{,}
                          \PY{n}{step}\PY{o}{=}\PY{l+s+s1}{\PYZsq{}}\PY{l+s+s1}{n}\PY{l+s+s1}{\PYZsq{}}\PY{p}{,}  
                          \PY{n}{selection}\PY{o}{=}\PY{l+s+s1}{\PYZsq{}}\PY{l+s+s1}{mu\PYZus{}plus\PYZus{}lambda}\PY{l+s+s1}{\PYZsq{}}\PY{p}{)}\PY{p}{[}\PY{l+m+mi}{1}\PY{p}{]}
          \PY{n}{hiper\PYZus{}n\PYZus{}plus} \PY{o}{=}  \PY{n}{p\PYZus{}exps}\PY{p}{(}\PY{l+s+s1}{\PYZsq{}}\PY{l+s+s1}{Hiper n   plus }\PY{l+s+s1}{\PYZsq{}}\PY{p}{,}\PY{n}{partial}\PY{p}{)}
\end{Verbatim}

    \section{Métricas de evaluación de
configuraciones}\label{muxe9tricas-de-evaluaciuxf3n-de-configuraciones}

Para justificar la conveniencia de una otra configuración de la
estrategia evolutiva utilizada para la búsqueda de los mínimos del
hiper-elipsoide rotado, primero tenemos que definir las funciones de
rendimiento con las que medir los resultados.

Lógicamente, estas funciones serán igualmente utilizables en el caso de
la función de Rastrigin.

    \begin{Verbatim}[commandchars=\\\{\}]
{\color{incolor}In [{\color{incolor}11}]:} \PY{k}{def} \PY{n+nf}{SR}\PY{p}{(}\PY{n}{results}\PY{p}{,}\PY{n}{f}\PY{p}{,}\PY{n}{epsilon}\PY{o}{=}\PY{l+m+mf}{1e\PYZhy{}6}\PY{p}{)}\PY{p}{:}
             \PY{l+s+sd}{\PYZdq{}\PYZdq{}\PYZdq{}}
         \PY{l+s+sd}{    Calcula el porcentaje de ejecuciones que }
         \PY{l+s+sd}{    acaban con éxito.}
         \PY{l+s+sd}{    El éxito se define como la situación en}
         \PY{l+s+sd}{    la que el fitness del mejor individuo }
         \PY{l+s+sd}{    dista del mínimo global en menos de epsilon.}
         
         \PY{l+s+sd}{    Args:}
         \PY{l+s+sd}{        results (array): un array cuyos elementos }
         \PY{l+s+sd}{            son arrays de individuos (el histórico}
         \PY{l+s+sd}{            de los mejores individuos de cada }
         \PY{l+s+sd}{            generación de una ejecución del }
         \PY{l+s+sd}{            algoritmo)}
         \PY{l+s+sd}{        f (function): la función que cuyos }
         \PY{l+s+sd}{            se buscan en la ejecución que }
         \PY{l+s+sd}{            generara los resultados de entrada}
         \PY{l+s+sd}{        epsilon (float): distancia máxima al }
         \PY{l+s+sd}{            mínimo global de la función que}
         \PY{l+s+sd}{            determina si una ejecución es }
         \PY{l+s+sd}{            considerada éxito o no.}
         \PY{l+s+sd}{            valor por defecto 1e\PYZhy{}5}
         \PY{l+s+sd}{            }
         \PY{l+s+sd}{    Returns:}
         \PY{l+s+sd}{        float: el porcentaje de ejecuciones que }
         \PY{l+s+sd}{            acaban con éxito}
         \PY{l+s+sd}{    \PYZdq{}\PYZdq{}\PYZdq{}}
             \PY{n}{best\PYZus{}inds}\PY{o}{=}\PY{p}{[}\PY{n}{hist}\PY{p}{[}\PY{o}{\PYZhy{}}\PY{l+m+mi}{1}\PY{p}{]} \PY{k}{for} \PY{n}{hist} \PY{o+ow}{in} \PY{n}{results}\PY{p}{]}
             \PY{n}{fitnesses}\PY{o}{=}\PY{p}{[}\PY{n}{f}\PY{p}{(}\PY{n}{ind}\PY{p}{[}\PY{l+s+s1}{\PYZsq{}}\PY{l+s+s1}{coords}\PY{l+s+s1}{\PYZsq{}}\PY{p}{]}\PY{p}{)} \PY{k}{for} \PY{n}{ind} \PY{o+ow}{in} \PY{n}{best\PYZus{}inds}\PY{p}{]}
             \PY{n}{successes}\PY{o}{=}\PY{n+nb}{filter}\PY{p}{(}\PY{k}{lambda} \PY{n}{x}\PY{p}{:} \PY{n}{x} \PY{o}{\PYZlt{}} \PY{n}{epsilon}\PY{p}{,} \PY{n}{fitnesses}\PY{p}{)}
             \PY{k}{return} \PY{n+nb}{float}\PY{p}{(}\PY{n+nb}{len}\PY{p}{(}\PY{n}{successes}\PY{p}{)}\PY{p}{)}\PY{o}{/}\PY{n+nb}{len}\PY{p}{(}\PY{n}{fitnesses}\PY{p}{)}
         
         
         \PY{k}{def} \PY{n+nf}{AES}\PY{p}{(}\PY{n}{results}\PY{p}{)}\PY{p}{:}
             \PY{l+s+sd}{\PYZdq{}\PYZdq{}\PYZdq{}}
         \PY{l+s+sd}{    Calcula el número medio de evaluaciones }
         \PY{l+s+sd}{    necesario hasta dar con la solución.}
         \PY{l+s+sd}{    Esta métrica tiene sentido dado que}
         \PY{l+s+sd}{    nuestra evaluaciones se dotan de }
         \PY{l+s+sd}{    condiciones de terminación sin }
         \PY{l+s+sd}{    necesariamente agotar el máximo }
         \PY{l+s+sd}{    de generaciones permitidas.}
         
         \PY{l+s+sd}{    Args:}
         \PY{l+s+sd}{        results (array): un array cuyos elementos }
         \PY{l+s+sd}{            son arrays de individuos (el histórico}
         \PY{l+s+sd}{            de los mejores individuos de cada }
         \PY{l+s+sd}{            generación de una ejecución del }
         \PY{l+s+sd}{            algoritmo)}
         \PY{l+s+sd}{            }
         \PY{l+s+sd}{    Returns:}
         \PY{l+s+sd}{        float: el número medio de evaluaciones }
         \PY{l+s+sd}{            hasta dar con la solución}
         \PY{l+s+sd}{    \PYZdq{}\PYZdq{}\PYZdq{}}
             \PY{k}{return} \PY{n+nb}{float}\PY{p}{(}\PY{n+nb}{sum}\PY{p}{(}\PY{p}{[}\PY{n+nb}{len}\PY{p}{(}\PY{n}{x}\PY{p}{)} \PY{k}{for} \PY{n}{x} \PY{o+ow}{in} \PY{n}{results}\PY{p}{]}\PY{p}{)}\PY{p}{)}\PY{o}{/}\PY{n+nb}{len}\PY{p}{(}\PY{n}{results}\PY{p}{)}
         
         
         \PY{k}{def} \PY{n+nf}{MBF}\PY{p}{(}\PY{n}{results}\PY{p}{,}\PY{n}{f}\PY{p}{)}\PY{p}{:}
             \PY{l+s+sd}{\PYZdq{}\PYZdq{}\PYZdq{}}
         \PY{l+s+sd}{    Calcula la media de los valores de fitness}
         \PY{l+s+sd}{    del mejor individuo encontrado (el individuo }
         \PY{l+s+sd}{    devuelto por el algoritmo) en un }
         \PY{l+s+sd}{    conjunto de ejecuciones del algoritmo }
         \PY{l+s+sd}{    evolutivo.}
         
         \PY{l+s+sd}{    Args:}
         \PY{l+s+sd}{        results (array): un array cuyos elementos }
         \PY{l+s+sd}{            son arrays de individuos (el histórico}
         \PY{l+s+sd}{            de los mejores individuos de cada }
         \PY{l+s+sd}{            generación de una ejecución del }
         \PY{l+s+sd}{            algoritmo)}
         \PY{l+s+sd}{        f (function): la función que cuyos }
         \PY{l+s+sd}{            se buscan en la ejecución que }
         \PY{l+s+sd}{            generara los resultados de entrada}
         \PY{l+s+sd}{            }
         \PY{l+s+sd}{    Returns:}
         \PY{l+s+sd}{        float: la media de los valores de fitness }
         \PY{l+s+sd}{            de los mejores individuos encontrados}
         \PY{l+s+sd}{            en varias ejecuciones del algortimo}
         \PY{l+s+sd}{    \PYZdq{}\PYZdq{}\PYZdq{}}
             \PY{n}{best\PYZus{}inds}\PY{o}{=}\PY{p}{[}\PY{n}{hist}\PY{p}{[}\PY{o}{\PYZhy{}}\PY{l+m+mi}{1}\PY{p}{]} \PY{k}{for} \PY{n}{hist} \PY{o+ow}{in} \PY{n}{results}\PY{p}{]}
             \PY{n}{fitnesses}\PY{o}{=}\PY{p}{[}\PY{n}{f}\PY{p}{(}\PY{n}{ind}\PY{p}{[}\PY{l+s+s1}{\PYZsq{}}\PY{l+s+s1}{coords}\PY{l+s+s1}{\PYZsq{}}\PY{p}{]}\PY{p}{)} \PY{k}{for} \PY{n}{ind} \PY{o+ow}{in} \PY{n}{best\PYZus{}inds}\PY{p}{]}
             \PY{k}{return} \PY{n+nb}{sum}\PY{p}{(}\PY{n}{fitnesses}\PY{p}{)}\PY{o}{/}\PY{n+nb}{len}\PY{p}{(}\PY{n}{fitnesses}\PY{p}{)}
\end{Verbatim}

    Por último, necesitamos también una función que dibuje la tabla de
resultados de un conjunto de experimentos y que funcione bién cuando
este documento se exporta a latex. Ésta es:

    \begin{Verbatim}[commandchars=\\\{\}]
{\color{incolor}In [{\color{incolor}12}]:} \PY{k+kn}{import} \PY{n+nn}{pandas} \PY{k+kn}{as} \PY{n+nn}{pd}
         \PY{n}{pd}\PY{o}{.}\PY{n}{set\PYZus{}option}\PY{p}{(}\PY{l+s+s1}{\PYZsq{}}\PY{l+s+s1}{display.notebook\PYZus{}repr\PYZus{}html}\PY{l+s+s1}{\PYZsq{}}\PY{p}{,} \PY{n+nb+bp}{True}\PY{p}{)}
         
         \PY{k}{def} \PY{n+nf}{\PYZus{}repr\PYZus{}latex\PYZus{}}\PY{p}{(}\PY{n+nb+bp}{self}\PY{p}{)}\PY{p}{:}
         \PY{c+c1}{\PYZsh{}     return \PYZdq{}\PYZbs{}centering\PYZob{}\PYZpc{}s\PYZcb{}\PYZdq{} \PYZpc{} self.to\PYZus{}latex()}
             \PY{k}{return} \PY{n+nb+bp}{self}\PY{o}{.}\PY{n}{to\PYZus{}latex}\PY{p}{(}\PY{p}{)}
         
         \PY{n}{pd}\PY{o}{.}\PY{n}{DataFrame}\PY{o}{.}\PY{n}{\PYZus{}repr\PYZus{}latex\PYZus{}} \PY{o}{=} \PY{n}{\PYZus{}repr\PYZus{}latex\PYZus{}}
         
         
         \PY{k}{def} \PY{n+nf}{render\PYZus{}metrics}\PY{p}{(}\PY{n}{metrics}\PY{p}{)}\PY{p}{:}
             \PY{l+s+sd}{\PYZdq{}\PYZdq{}\PYZdq{}}
         \PY{l+s+sd}{    Dibuja de forma tabulada las métricas de }
         \PY{l+s+sd}{    rendimiento de una lista de experimentos.}
         \PY{l+s+sd}{    }
         \PY{l+s+sd}{    Args:}
         \PY{l+s+sd}{        metrics (array): array donde cada }
         \PY{l+s+sd}{            elemento es un array compuesto de}
         \PY{l+s+sd}{            [\PYZsq{}nombre del experimento\PYZsq{},}
         \PY{l+s+sd}{             SR del exp,}
         \PY{l+s+sd}{             AES del exp, ,}
         \PY{l+s+sd}{             MBF del exp]}
         \PY{l+s+sd}{    \PYZdq{}\PYZdq{}\PYZdq{}}    
             \PY{n}{data} \PY{o}{=} \PY{p}{\PYZob{}}\PY{l+s+s1}{\PYZsq{}}\PY{l+s+s1}{SR}\PY{l+s+s1}{\PYZsq{}} \PY{p}{:} \PY{p}{[}\PY{n}{x}\PY{p}{[}\PY{l+m+mi}{1}\PY{p}{]} \PY{k}{for} \PY{n}{x} \PY{o+ow}{in} \PY{n}{metrics}\PY{p}{]}\PY{p}{,}
                     \PY{l+s+s1}{\PYZsq{}}\PY{l+s+s1}{AES}\PY{l+s+s1}{\PYZsq{}}\PY{p}{:} \PY{p}{[}\PY{n}{x}\PY{p}{[}\PY{l+m+mi}{2}\PY{p}{]} \PY{k}{for} \PY{n}{x} \PY{o+ow}{in} \PY{n}{metrics}\PY{p}{]}\PY{p}{,}
                     \PY{l+s+s1}{\PYZsq{}}\PY{l+s+s1}{MBF}\PY{l+s+s1}{\PYZsq{}}\PY{p}{:} \PY{p}{[}\PY{n}{x}\PY{p}{[}\PY{l+m+mi}{3}\PY{p}{]} \PY{k}{for} \PY{n}{x} \PY{o+ow}{in} \PY{n}{metrics}\PY{p}{]}\PY{p}{\PYZcb{}} 
             \PY{k}{return} \PY{n}{pd}\PY{o}{.}\PY{n}{DataFrame}\PY{p}{(}\PY{n}{data}\PY{p}{,} 
                                 \PY{n}{index}\PY{o}{=}\PY{p}{[}\PY{n}{x}\PY{p}{[}\PY{l+m+mi}{0}\PY{p}{]} \PY{k}{for} \PY{n}{x} \PY{o+ow}{in} \PY{n}{metrics}\PY{p}{]}\PY{p}{,}
                                 \PY{n}{columns}\PY{o}{=}\PY{p}{[}\PY{l+s+s1}{\PYZsq{}}\PY{l+s+s1}{SR}\PY{l+s+s1}{\PYZsq{}}\PY{p}{,}\PY{l+s+s1}{\PYZsq{}}\PY{l+s+s1}{AES}\PY{l+s+s1}{\PYZsq{}}\PY{p}{,}\PY{l+s+s1}{\PYZsq{}}\PY{l+s+s1}{MBF}\PY{l+s+s1}{\PYZsq{}}\PY{p}{]}\PY{p}{)}
         
         
         \PY{k}{def} \PY{n+nf}{render\PYZus{}experiments}\PY{p}{(}\PY{n}{experiments}\PY{p}{,}\PY{n}{f}\PY{p}{,}\PY{n}{epsilon}\PY{p}{)}\PY{p}{:}
             \PY{l+s+sd}{\PYZdq{}\PYZdq{}\PYZdq{}}
         \PY{l+s+sd}{    Dibuja de forma tabulada un diccionario}
         \PY{l+s+sd}{    de experimentos.}
         \PY{l+s+sd}{    \PYZdq{}\PYZdq{}\PYZdq{}}    
             \PY{n}{metrics} \PY{o}{=} \PY{p}{[}\PY{p}{[}\PY{n}{k}\PY{p}{,} \PY{n}{SR}\PY{p}{(}\PY{n}{v}\PY{p}{,} \PY{n}{f}\PY{p}{,} \PY{n}{epsilon}\PY{o}{=}\PY{n}{epsilon}\PY{p}{)}\PY{p}{,}
                         \PY{n}{AES}\PY{p}{(}\PY{n}{v}\PY{p}{)}\PY{p}{,} \PY{n}{MBF}\PY{p}{(}\PY{n}{v}\PY{p}{,} \PY{n}{f}\PY{p}{)}\PY{p}{]}
                        \PY{k}{for} \PY{n}{k}\PY{p}{,}\PY{n}{v} \PY{o+ow}{in} \PY{n}{experiments}\PY{o}{.}\PY{n}{iteritems}\PY{p}{(}\PY{p}{)}\PY{p}{]}    
             \PY{k}{return} \PY{n}{render\PYZus{}metrics}\PY{p}{(}\PY{n}{metrics}\PY{p}{)}
         
         \PY{k}{def} \PY{n+nf}{merge\PYZus{}dicts}\PY{p}{(}\PY{o}{*}\PY{n}{dict\PYZus{}args}\PY{p}{)}\PY{p}{:}
             \PY{l+s+sd}{\PYZdq{}\PYZdq{}\PYZdq{}}
         \PY{l+s+sd}{    Given any number of dicts, shallow copy and merge into a new dict,}
         \PY{l+s+sd}{    precedence goes to key value pairs in latter dicts.}
         \PY{l+s+sd}{    \PYZdq{}\PYZdq{}\PYZdq{}}
             \PY{n}{result} \PY{o}{=} \PY{p}{\PYZob{}}\PY{p}{\PYZcb{}}
             \PY{k}{for} \PY{n}{dictionary} \PY{o+ow}{in} \PY{n}{dict\PYZus{}args}\PY{p}{:}
                 \PY{n}{result}\PY{o}{.}\PY{n}{update}\PY{p}{(}\PY{n}{dictionary}\PY{p}{)}
             \PY{k}{return} \PY{n}{result}
\end{Verbatim}

    \section{Ejercicio 7: Resultados para la función hiper-elipsoide
rotado}\label{ejercicio-7-resultados-para-la-funciuxf3n-hiper-elipsoide-rotado}

Una vez disponemos de las funciones anteriores, podemos calcular y
mostrar la tabla de resultados para la búsqueda del mínimo global del
hiper-elipsoide rotado de 10 dimensiones, tomando 30 mus y 200 lambdas
en las distintas configuraciones que hemos probado.

    \begin{Verbatim}[commandchars=\\\{\}]
{\color{incolor}In [{\color{incolor}105}]:} \PY{n}{render\PYZus{}experiments}\PY{p}{(}\PY{n}{merge\PYZus{}dicts}\PY{p}{(}\PY{n}{hiper\PYZus{}one\PYZus{}comma}\PY{p}{,}
                                         \PY{n}{hiper\PYZus{}n\PYZus{}comma}\PY{p}{,}
                                         \PY{n}{hiper\PYZus{}one\PYZus{}plus}\PY{p}{,}
                                         \PY{n}{hiper\PYZus{}n\PYZus{}plus}\PY{p}{)}\PY{p}{,} 
                             \PY{n}{hiper\PYZus{}elipsoide\PYZus{}rotado}\PY{p}{,} \PY{l+m+mf}{1e\PYZhy{}10}\PY{p}{)}
\end{Verbatim}
\texttt{\color{outcolor}Out[{\color{outcolor}105}]:}
    
    \begin{tabular}{lrrr}
\toprule
{} &   SR &     AES &            MBF \\
\midrule
Hiper one plus  &  1.0 &  1000.0 &   1.105618e-77 \\
Hiper one comma &  1.0 &  1000.0 &   3.186863e-64 \\
Hiper n   comma &  1.0 &  1000.0 &  6.218410e-233 \\
Hiper n   plus  &  1.0 &  1000.0 &  7.882819e-241 \\
\bottomrule
\end{tabular}

    

    \[ \] Los resultados son en general muy satisfactorios. Todos se quedan
a una distancia menor de 1e-10 del mínimo. En realidad, a menos de
4e-64.

\textbf{Se observa claramente que la estrategia de n pasos profundiza
mucho más en la solución.} En parte, es de entender, al tratarse de una
función con un único mínimo.

Dado que como veremos más tarde, la búsqueda del óptimo de la función de
Rastrigin ha sido bastante más complicada, prefiero retrasar hasta los
comentarios de ésta las distintas experimentaciones que he hecho para
intentar ajustar al máximo los parámetros de la estrategia. O dicho de
otro modo, para la función hiper-elipsoide rotado los resultados son tan
buenos que prefiero centrar todos los esfuerzos en la función de
Rastrigin.

Si acaso, para comprobar que efectivamente indicando un error de
terminación, la convergencia se produce en pocas generaciones, podemos
verlo con:

    \begin{Verbatim}[commandchars=\\\{\}]
{\color{incolor}In [{\color{incolor}95}]:} \PY{k}{def} \PY{n+nf}{partial}\PY{p}{(}\PY{n}{i}\PY{p}{)}\PY{p}{:}
             \PY{k}{return} \PY{n}{live}\PY{p}{(}\PY{p}{[}\PY{o}{\PYZhy{}}\PY{l+m+mf}{65.54} \PY{p}{,} \PY{l+m+mf}{65.54}\PY{p}{]}\PY{p}{,}\PY{l+m+mi}{10}\PY{p}{,}
                         \PY{n}{hiper\PYZus{}elipsoide\PYZus{}rotado}\PY{p}{,}
                         \PY{n}{step}\PY{o}{=}\PY{l+s+s1}{\PYZsq{}}\PY{l+s+s1}{one}\PY{l+s+s1}{\PYZsq{}}\PY{p}{,}  
                         \PY{n}{selection}\PY{o}{=}\PY{l+s+s1}{\PYZsq{}}\PY{l+s+s1}{mu\PYZus{}comma\PYZus{}lambda}\PY{l+s+s1}{\PYZsq{}}\PY{p}{,}
                         \PY{n}{termination\PYZus{}delta}\PY{o}{=}\PY{l+m+mf}{1e\PYZhy{}10}\PY{p}{)}\PY{p}{[}\PY{l+m+mi}{1}\PY{p}{]}
         \PY{n}{hiper\PYZus{}one\PYZus{}comma\PYZus{}delta} \PY{o}{=}  \PY{n}{p\PYZus{}exps}\PY{p}{(}\PY{l+s+s1}{\PYZsq{}}\PY{l+s+s1}{Hiper one comma delta }\PY{l+s+s1}{\PYZsq{}}\PY{p}{,}\PY{n}{partial}\PY{p}{)}
         
         \PY{k}{def} \PY{n+nf}{partial}\PY{p}{(}\PY{n}{i}\PY{p}{)}\PY{p}{:}
             \PY{k}{return} \PY{n}{live}\PY{p}{(}\PY{p}{[}\PY{o}{\PYZhy{}}\PY{l+m+mf}{65.54} \PY{p}{,} \PY{l+m+mf}{65.54}\PY{p}{]}\PY{p}{,}\PY{l+m+mi}{10}\PY{p}{,}
                         \PY{n}{hiper\PYZus{}elipsoide\PYZus{}rotado}\PY{p}{,}
                         \PY{n}{step}\PY{o}{=}\PY{l+s+s1}{\PYZsq{}}\PY{l+s+s1}{n}\PY{l+s+s1}{\PYZsq{}}\PY{p}{,}  
                         \PY{n}{selection}\PY{o}{=}\PY{l+s+s1}{\PYZsq{}}\PY{l+s+s1}{mu\PYZus{}comma\PYZus{}lambda}\PY{l+s+s1}{\PYZsq{}}\PY{p}{,}
                         \PY{n}{termination\PYZus{}delta}\PY{o}{=}\PY{l+m+mf}{1e\PYZhy{}10}\PY{p}{)}\PY{p}{[}\PY{l+m+mi}{1}\PY{p}{]}
         \PY{n}{hiper\PYZus{}n\PYZus{}comma\PYZus{}delta} \PY{o}{=}  \PY{n}{p\PYZus{}exps}\PY{p}{(}\PY{l+s+s1}{\PYZsq{}}\PY{l+s+s1}{Hiper n   comma delta }\PY{l+s+s1}{\PYZsq{}}\PY{p}{,}\PY{n}{partial}\PY{p}{)}
         
         \PY{k}{def} \PY{n+nf}{partial}\PY{p}{(}\PY{n}{i}\PY{p}{)}\PY{p}{:}
             \PY{k}{return} \PY{n}{live}\PY{p}{(}\PY{p}{[}\PY{o}{\PYZhy{}}\PY{l+m+mf}{65.54} \PY{p}{,} \PY{l+m+mf}{65.54}\PY{p}{]}\PY{p}{,}\PY{l+m+mi}{10}\PY{p}{,}
                         \PY{n}{hiper\PYZus{}elipsoide\PYZus{}rotado}\PY{p}{,}
                         \PY{n}{step}\PY{o}{=}\PY{l+s+s1}{\PYZsq{}}\PY{l+s+s1}{one}\PY{l+s+s1}{\PYZsq{}}\PY{p}{,}  
                         \PY{n}{selection}\PY{o}{=}\PY{l+s+s1}{\PYZsq{}}\PY{l+s+s1}{mu\PYZus{}plus\PYZus{}lambda}\PY{l+s+s1}{\PYZsq{}}\PY{p}{,}
                         \PY{n}{termination\PYZus{}delta}\PY{o}{=}\PY{l+m+mf}{1e\PYZhy{}10}\PY{p}{)}\PY{p}{[}\PY{l+m+mi}{1}\PY{p}{]}
         \PY{n}{hiper\PYZus{}one\PYZus{}plus\PYZus{}delta} \PY{o}{=}  \PY{n}{p\PYZus{}exps}\PY{p}{(}\PY{l+s+s1}{\PYZsq{}}\PY{l+s+s1}{Hiper one plus  delta }\PY{l+s+s1}{\PYZsq{}}\PY{p}{,}\PY{n}{partial}\PY{p}{)}
         
         \PY{k}{def} \PY{n+nf}{partial}\PY{p}{(}\PY{n}{i}\PY{p}{)}\PY{p}{:}
             \PY{k}{return} \PY{n}{live}\PY{p}{(}\PY{p}{[}\PY{o}{\PYZhy{}}\PY{l+m+mf}{65.54} \PY{p}{,} \PY{l+m+mf}{65.54}\PY{p}{]}\PY{p}{,}\PY{l+m+mi}{10}\PY{p}{,}
                         \PY{n}{hiper\PYZus{}elipsoide\PYZus{}rotado}\PY{p}{,}
                         \PY{n}{step}\PY{o}{=}\PY{l+s+s1}{\PYZsq{}}\PY{l+s+s1}{n}\PY{l+s+s1}{\PYZsq{}}\PY{p}{,}  
                         \PY{n}{selection}\PY{o}{=}\PY{l+s+s1}{\PYZsq{}}\PY{l+s+s1}{mu\PYZus{}plus\PYZus{}lambda}\PY{l+s+s1}{\PYZsq{}}\PY{p}{,}
                         \PY{n}{termination\PYZus{}delta}\PY{o}{=}\PY{l+m+mf}{1e\PYZhy{}10}\PY{p}{)}\PY{p}{[}\PY{l+m+mi}{1}\PY{p}{]}
         \PY{n}{hiper\PYZus{}n\PYZus{}plus\PYZus{}delta} \PY{o}{=}  \PY{n}{p\PYZus{}exps}\PY{p}{(}\PY{l+s+s1}{\PYZsq{}}\PY{l+s+s1}{Hiper n   comma delta }\PY{l+s+s1}{\PYZsq{}}\PY{p}{,}\PY{n}{partial}\PY{p}{)} 
         
         \PY{n}{render\PYZus{}experiments}\PY{p}{(}\PY{n}{merge\PYZus{}dicts}\PY{p}{(}\PY{n}{hiper\PYZus{}one\PYZus{}comma\PYZus{}delta}\PY{p}{,}
                                        \PY{n}{hiper\PYZus{}n\PYZus{}comma\PYZus{}delta}\PY{p}{,}
                                        \PY{n}{hiper\PYZus{}one\PYZus{}plus\PYZus{}delta}\PY{p}{,}
                                        \PY{n}{hiper\PYZus{}n\PYZus{}plus\PYZus{}delta}\PY{p}{)}\PY{p}{,} 
                            \PY{n}{hiper\PYZus{}elipsoide\PYZus{}rotado}\PY{p}{,} \PY{l+m+mf}{1e\PYZhy{}10}\PY{p}{)}
\end{Verbatim}
\texttt{\color{outcolor}Out[{\color{outcolor}95}]:}
    
    \begin{tabular}{lrrr}
\toprule
{} &   SR &     AES &           MBF \\
\midrule
Hiper n   comma delta &  1.0 &   79.10 &  7.006832e-11 \\
Hiper one comma delta &  1.0 &  165.15 &  9.256651e-11 \\
Hiper one plus  delta &  1.0 &  134.10 &  8.620310e-11 \\
Hiper n   plus  delta &  1.0 &   84.65 &  6.995538e-11 \\
\bottomrule
\end{tabular}

    

    \[ \] Nuevamente comprobamos que la estrategia de n pasos es más rápida
y que en todos los experimentos se encuentra una solucíon que satisfaga
el delta de terminación (error mínimo). Es decir, la tasa de éxito es
del 100\% en todos los casos.

    \section{Ejercicio 9: Análisis equivalente para la función de
Rastrigin}\label{ejercicio-9-anuxe1lisis-equivalente-para-la-funciuxf3n-de-rastrigin}

Ya que tenemos todas las herramientas necesarias, basta simplemente con
realizar los experimentos y mostrar los resultados:

    \begin{Verbatim}[commandchars=\\\{\}]
{\color{incolor}In [{\color{incolor}106}]:} \PY{k}{def} \PY{n+nf}{partial}\PY{p}{(}\PY{n}{i}\PY{p}{)}\PY{p}{:}
              \PY{k}{return} \PY{n}{live}\PY{p}{(}\PY{p}{[}\PY{o}{\PYZhy{}}\PY{l+m+mf}{5.12}\PY{p}{,}\PY{l+m+mf}{5.12}\PY{p}{]}\PY{p}{,}\PY{l+m+mi}{10}\PY{p}{,}
                          \PY{n}{rastrigin}\PY{p}{,}
                          \PY{n}{step}\PY{o}{=}\PY{l+s+s1}{\PYZsq{}}\PY{l+s+s1}{one}\PY{l+s+s1}{\PYZsq{}}\PY{p}{,}  
                          \PY{n}{selection}\PY{o}{=}\PY{l+s+s1}{\PYZsq{}}\PY{l+s+s1}{mu\PYZus{}comma\PYZus{}lambda}\PY{l+s+s1}{\PYZsq{}}\PY{p}{)}\PY{p}{[}\PY{l+m+mi}{1}\PY{p}{]}
          \PY{n}{rast\PYZus{}one\PYZus{}comma} \PY{o}{=}  \PY{n}{p\PYZus{}exps}\PY{p}{(}\PY{l+s+s1}{\PYZsq{}}\PY{l+s+s1}{Rastrigin one comma }\PY{l+s+s1}{\PYZsq{}}\PY{p}{,}\PY{n}{partial}\PY{p}{)}
          
          \PY{k}{def} \PY{n+nf}{partial}\PY{p}{(}\PY{n}{i}\PY{p}{)}\PY{p}{:}
              \PY{k}{return} \PY{n}{live}\PY{p}{(}\PY{p}{[}\PY{o}{\PYZhy{}}\PY{l+m+mf}{5.12}\PY{p}{,}\PY{l+m+mf}{5.12}\PY{p}{]}\PY{p}{,}\PY{l+m+mi}{10}\PY{p}{,}
                          \PY{n}{rastrigin}\PY{p}{,}
                          \PY{n}{step}\PY{o}{=}\PY{l+s+s1}{\PYZsq{}}\PY{l+s+s1}{n}\PY{l+s+s1}{\PYZsq{}}\PY{p}{,}  
                          \PY{n}{selection}\PY{o}{=}\PY{l+s+s1}{\PYZsq{}}\PY{l+s+s1}{mu\PYZus{}comma\PYZus{}lambda}\PY{l+s+s1}{\PYZsq{}}\PY{p}{)}\PY{p}{[}\PY{l+m+mi}{1}\PY{p}{]}
          \PY{n}{rast\PYZus{}n\PYZus{}comma} \PY{o}{=}  \PY{n}{p\PYZus{}exps}\PY{p}{(}\PY{l+s+s1}{\PYZsq{}}\PY{l+s+s1}{Rastrigin n   comma }\PY{l+s+s1}{\PYZsq{}}\PY{p}{,}\PY{n}{partial}\PY{p}{)}
          
          \PY{k}{def} \PY{n+nf}{partial}\PY{p}{(}\PY{n}{i}\PY{p}{)}\PY{p}{:}
              \PY{k}{return} \PY{n}{live}\PY{p}{(}\PY{p}{[}\PY{o}{\PYZhy{}}\PY{l+m+mf}{5.12}\PY{p}{,}\PY{l+m+mf}{5.12}\PY{p}{]}\PY{p}{,}\PY{l+m+mi}{10}\PY{p}{,}
                          \PY{n}{rastrigin}\PY{p}{,}
                          \PY{n}{step}\PY{o}{=}\PY{l+s+s1}{\PYZsq{}}\PY{l+s+s1}{one}\PY{l+s+s1}{\PYZsq{}}\PY{p}{,}  
                          \PY{n}{selection}\PY{o}{=}\PY{l+s+s1}{\PYZsq{}}\PY{l+s+s1}{mu\PYZus{}plus\PYZus{}lambda}\PY{l+s+s1}{\PYZsq{}}\PY{p}{)}\PY{p}{[}\PY{l+m+mi}{1}\PY{p}{]}
          \PY{n}{rast\PYZus{}one\PYZus{}plus} \PY{o}{=}  \PY{n}{p\PYZus{}exps}\PY{p}{(}\PY{l+s+s1}{\PYZsq{}}\PY{l+s+s1}{Rastrigin one plus }\PY{l+s+s1}{\PYZsq{}}\PY{p}{,}\PY{n}{partial}\PY{p}{)}
          
          \PY{k}{def} \PY{n+nf}{partial}\PY{p}{(}\PY{n}{i}\PY{p}{)}\PY{p}{:}
              \PY{k}{return} \PY{n}{live}\PY{p}{(}\PY{p}{[}\PY{o}{\PYZhy{}}\PY{l+m+mf}{5.12}\PY{p}{,}\PY{l+m+mf}{5.12}\PY{p}{]}\PY{p}{,}\PY{l+m+mi}{10}\PY{p}{,}
                          \PY{n}{rastrigin}\PY{p}{,}
                          \PY{n}{step}\PY{o}{=}\PY{l+s+s1}{\PYZsq{}}\PY{l+s+s1}{n}\PY{l+s+s1}{\PYZsq{}}\PY{p}{,}  
                          \PY{n}{selection}\PY{o}{=}\PY{l+s+s1}{\PYZsq{}}\PY{l+s+s1}{mu\PYZus{}plus\PYZus{}lambda}\PY{l+s+s1}{\PYZsq{}}\PY{p}{)}\PY{p}{[}\PY{l+m+mi}{1}\PY{p}{]}
          \PY{n}{rast\PYZus{}n\PYZus{}plus} \PY{o}{=}  \PY{n}{p\PYZus{}exps}\PY{p}{(}\PY{l+s+s1}{\PYZsq{}}\PY{l+s+s1}{Rastrigin n   plus }\PY{l+s+s1}{\PYZsq{}}\PY{p}{,}\PY{n}{partial}\PY{p}{)} 
          
          \PY{n}{render\PYZus{}experiments}\PY{p}{(}\PY{n}{merge\PYZus{}dicts}\PY{p}{(}\PY{n}{rast\PYZus{}one\PYZus{}comma}\PY{p}{,}
                                         \PY{n}{rast\PYZus{}n\PYZus{}comma}\PY{p}{,}
                                         \PY{n}{rast\PYZus{}one\PYZus{}plus}\PY{p}{,}
                                         \PY{n}{rast\PYZus{}n\PYZus{}plus}\PY{p}{)}\PY{p}{,} 
                             \PY{n}{rastrigin}\PY{p}{,} \PY{l+m+mf}{1e\PYZhy{}10}\PY{p}{)}
\end{Verbatim}
\texttt{\color{outcolor}Out[{\color{outcolor}106}]:}
    
    \begin{tabular}{lrrr}
\toprule
{} &   SR &     AES &       MBF \\
\midrule
Rastrigin one comma  &  0.3 &  1000.0 &  1.542187 \\
Rastrigin n   plus   &  0.0 &  1000.0 &  3.333113 \\
Rastrigin n   comma  &  0.1 &  1000.0 &  1.840674 \\
Rastrigin one plus   &  0.0 &  1000.0 &  2.686389 \\
\bottomrule
\end{tabular}

    

    \[ \] En el caso de Rastrigin vemos que los resultados son en general
decepcionantes con los valores por defecto de los parámetros del
algoritmo.

Las estrategias que descartan los mus en la selección ofrecen mejor
rendimiento, pero en general los rendimientos son malos.

De hecho, la media de distancias al mínimo global no baja de 1.54, cosa
decepcionante por completo.

He hecho miles de experimentos cambiando uno a uno todos los posibles
parámetros de la estrategia: sigmas, valor mínimo de sigma, taus, mus,
lambdas\ldots{}

\section{Ejercicio 8: Búsqueda de la mejor configuración de
parámetros}\label{ejercicio-8-buxfasqueda-de-la-mejor-configuraciuxf3n-de-paruxe1metros}

\textbf{A continuación muestro el modo de búsqueda de la mejor
configuración de parámetros que he seguido para la función de
Rastrigin.} Con dije antes, en el caso del hiperelipsoide rotado, los
resultados obtenidos con los valores por defecto son tan buenos que he
decidido no proceder a tal análisis.

Básicamente, he preparado un script que realiza una batería de pruebas
explorando todas las combinaciones de paso, selección y combinación y
probando con factores de multiplicación de taus y taus primas 0.25, 0.5,
1, 2 y 4.

Obsérvese que el ataque por fuerza siguiente para descubrir la mejor
parametrización lo limito a sólo 250 generaciones. La razón es que
después de acabar esta práctica me gustaría seguir conservando el
ordenador en el que la realizo y todas sus CPUs. Por favor, disculpe la
libertad que me he tomado en esta limitación.

    \begin{Verbatim}[commandchars=\\\{\}]
{\color{incolor}In [{\color{incolor}43}]:} \PY{k+kn}{import} \PY{n+nn}{itertools} \PY{k+kn}{as} \PY{n+nn}{it}
         
         \PY{k}{def} \PY{n+nf}{p\PYZus{}live}\PY{p}{(}\PY{n}{step}\PY{p}{,}\PY{n}{s}\PY{p}{,}\PY{n}{r}\PY{p}{,}\PY{n}{t}\PY{p}{,}\PY{n}{t\PYZus{}p}\PY{p}{,}\PY{n}{i}\PY{p}{)}\PY{p}{:}
             \PY{k}{return} \PY{n}{live}\PY{p}{(}\PY{p}{[}\PY{o}{\PYZhy{}}\PY{l+m+mf}{5.12}\PY{p}{,}\PY{l+m+mf}{5.12}\PY{p}{]}\PY{p}{,}
                         \PY{l+m+mi}{10}\PY{p}{,}
                         \PY{n}{rastrigin}\PY{p}{,}
                         \PY{n}{step}\PY{o}{=}\PY{n}{step}\PY{p}{,}
                         \PY{n}{selection}\PY{o}{=}\PY{n}{s}\PY{p}{,}
                         \PY{n}{recombination}\PY{o}{=}\PY{n}{r}\PY{p}{,}
                         \PY{n}{tau\PYZus{}factor}\PY{o}{=}\PY{n}{t}\PY{p}{,}
                         \PY{n}{tau\PYZus{}prima\PYZus{}factor}\PY{o}{=}\PY{n}{t\PYZus{}p}\PY{p}{,}
                         \PY{n}{max\PYZus{}generations}\PY{o}{=}\PY{l+m+mi}{250}\PY{p}{)}\PY{p}{[}\PY{l+m+mi}{1}\PY{p}{]}
         
         \PY{k}{def} \PY{n+nf}{do\PYZus{}exploration}\PY{p}{(}\PY{p}{)}\PY{p}{:}
             \PY{n}{result} \PY{o}{=} \PY{p}{[}\PY{p}{]}
             \PY{n}{sels} \PY{o}{=} \PY{p}{[}\PY{l+s+s1}{\PYZsq{}}\PY{l+s+s1}{mu\PYZus{}comma\PYZus{}lambda}\PY{l+s+s1}{\PYZsq{}}\PY{p}{,}\PY{l+s+s1}{\PYZsq{}}\PY{l+s+s1}{mu\PYZus{}plus\PYZus{}lambda}\PY{l+s+s1}{\PYZsq{}}\PY{p}{]}
             \PY{n}{recs} \PY{o}{=} \PY{p}{[}\PY{l+s+s1}{\PYZsq{}}\PY{l+s+s1}{intermediate}\PY{l+s+s1}{\PYZsq{}}\PY{p}{,}\PY{l+s+s1}{\PYZsq{}}\PY{l+s+s1}{discrete}\PY{l+s+s1}{\PYZsq{}}\PY{p}{]}
             \PY{n}{taus} \PY{o}{=} \PY{p}{[}\PY{l+m+mf}{0.25}\PY{p}{,} \PY{l+m+mf}{0.5}\PY{p}{,} \PY{l+m+mi}{1}\PY{p}{,} \PY{l+m+mi}{2}\PY{p}{,} \PY{l+m+mi}{4}\PY{p}{]}
             \PY{n}{taus\PYZus{}p} \PY{o}{=} \PY{p}{[}\PY{l+m+mf}{0.25}\PY{p}{,} \PY{l+m+mf}{0.5}\PY{p}{,} \PY{l+m+mi}{1}\PY{p}{,} \PY{l+m+mi}{2}\PY{p}{,} \PY{l+m+mi}{4}\PY{p}{]}
             \PY{n}{template} \PY{o}{=} \PY{l+s+s1}{\PYZsq{}}\PY{l+s+s1}{Rast n \PYZob{}\PYZcb{} \PYZob{}\PYZcb{} tau=\PYZob{}\PYZcb{} tau prima=\PYZob{}\PYZcb{}}\PY{l+s+s1}{\PYZsq{}}
             \PY{k}{for} \PY{n}{s}\PY{p}{,}\PY{n}{r}\PY{p}{,}\PY{n}{t}\PY{p}{,}\PY{n}{t\PYZus{}p} \PY{o+ow}{in} \PY{n}{it}\PY{o}{.}\PY{n}{product}\PY{p}{(}\PY{n}{sels}\PY{p}{,}\PY{n}{recs}\PY{p}{,}\PY{n}{taus}\PY{p}{,}\PY{n}{taus\PYZus{}p}\PY{p}{)}\PY{p}{:}              
                 \PY{n}{name} \PY{o}{=} \PY{n}{template}\PY{o}{.}\PY{n}{format}\PY{p}{(}\PY{n}{r}\PY{p}{,}\PY{n}{s}\PY{p}{,}\PY{n}{t}\PY{p}{,}\PY{n}{t\PYZus{}p}\PY{p}{)}        
                 \PY{k}{print} \PY{n}{name}
                 \PY{n}{exps} \PY{o}{=} \PY{n}{p\PYZus{}exps}\PY{p}{(}\PY{n}{name}\PY{p}{,}\PY{n}{partial}\PY{p}{(}\PY{n}{p\PYZus{}live}\PY{p}{,}\PY{l+s+s1}{\PYZsq{}}\PY{l+s+s1}{n}\PY{l+s+s1}{\PYZsq{}}\PY{p}{,}\PY{n}{s}\PY{p}{,}\PY{n}{r}\PY{p}{,}\PY{n}{t}\PY{p}{,}\PY{n}{t\PYZus{}p}\PY{p}{)}\PY{p}{,}\PY{n}{n}\PY{o}{=}\PY{l+m+mi}{20}\PY{p}{)}
                 \PY{n}{metrics} \PY{o}{=} \PY{n}{render\PYZus{}experiments}\PY{p}{(}\PY{n}{exps}\PY{p}{,}\PY{n}{rastrigin}\PY{p}{,}\PY{n}{epsilon}\PY{o}{=}\PY{l+m+mf}{0.1}\PY{p}{)}  
                 \PY{n}{result} \PY{o}{+}\PY{o}{=} \PY{p}{[}\PY{n}{metrics}\PY{p}{]}
                 
             \PY{n}{template} \PY{o}{=} \PY{l+s+s1}{\PYZsq{}}\PY{l+s+s1}{Rast one \PYZob{}\PYZcb{} \PYZob{}\PYZcb{} tau=\PYZob{}\PYZcb{}}\PY{l+s+s1}{\PYZsq{}}
             \PY{k}{for} \PY{n}{s}\PY{p}{,}\PY{n}{r}\PY{p}{,}\PY{n}{t} \PY{o+ow}{in} \PY{n}{it}\PY{o}{.}\PY{n}{product}\PY{p}{(}\PY{n}{sels}\PY{p}{,}\PY{n}{recs}\PY{p}{,}\PY{n}{taus}\PY{p}{)}\PY{p}{:}        
                 \PY{n}{name} \PY{o}{=} \PY{n}{template}\PY{o}{.}\PY{n}{format}\PY{p}{(}\PY{n}{r}\PY{p}{,}\PY{n}{s}\PY{p}{,}\PY{n}{t}\PY{p}{)}       
                 \PY{k}{print} \PY{n}{name} 
                 \PY{n}{exps} \PY{o}{=} \PY{n}{p\PYZus{}exps}\PY{p}{(}\PY{n}{name}\PY{p}{,}\PY{n}{partial}\PY{p}{(}\PY{n}{p\PYZus{}live}\PY{p}{,}\PY{l+s+s1}{\PYZsq{}}\PY{l+s+s1}{one}\PY{l+s+s1}{\PYZsq{}}\PY{p}{,}\PY{n}{s}\PY{p}{,}\PY{n}{r}\PY{p}{,}\PY{n}{t}\PY{p}{,}\PY{l+m+mi}{0}\PY{p}{)}\PY{p}{,}\PY{n}{n}\PY{o}{=}\PY{l+m+mi}{20}\PY{p}{)}
                 \PY{n}{metrics} \PY{o}{=} \PY{n}{render\PYZus{}experiments}\PY{p}{(}\PY{n}{exps}\PY{p}{,}\PY{n}{rastrigin}\PY{p}{,}\PY{n}{epsilon}\PY{o}{=}\PY{l+m+mf}{0.1}\PY{p}{)}  
                 \PY{n}{result} \PY{o}{+}\PY{o}{=} \PY{p}{[}\PY{n}{metrics}\PY{p}{]}
             \PY{k}{return} \PY{n}{result}
         
                 
         \PY{n}{exploration} \PY{o}{=} \PY{n}{do\PYZus{}exploration}\PY{p}{(}\PY{p}{)}    
         \PY{n}{pd}\PY{o}{.}\PY{n}{set\PYZus{}option}\PY{p}{(}\PY{l+s+s1}{\PYZsq{}}\PY{l+s+s1}{display.max\PYZus{}rows}\PY{l+s+s1}{\PYZsq{}}\PY{p}{,} \PY{l+m+mi}{500}\PY{p}{)}
         \PY{n}{result} \PY{o}{=} \PY{n}{pd}\PY{o}{.}\PY{n}{concat}\PY{p}{(}\PY{n}{exploration}\PY{p}{)}    
         \PY{n}{result}\PY{o}{.}\PY{n}{sort\PYZus{}values}\PY{p}{(}\PY{l+s+s1}{\PYZsq{}}\PY{l+s+s1}{SR}\PY{l+s+s1}{\PYZsq{}}\PY{p}{)}
\end{Verbatim}
\texttt{\color{outcolor}Out[{\color{outcolor}43}]:}
    
    \begin{tabular}{lrrr}
\toprule
{} &    SR &    AES &        MBF \\
\midrule
Rast n inter plus  tau=0.5 tau prima=4     &  0.00 &  250.0 &   4.328072 \\
Rast n disc  plus  tau=0.25 tau prima=2    &  0.00 &  250.0 &   4.129081 \\
Rast n disc  plus  tau=0.25 tau prima=1    &  0.00 &  250.0 &   3.731096 \\
Rast n disc  plus  tau=0.25 tau prima=0.25 &  0.00 &  250.0 &  28.410628 \\
Rast n inter plus  tau=4 tau prima=4       &  0.00 &  250.0 &   3.830592 \\
Rast n inter plus  tau=4 tau prima=2       &  0.00 &  250.0 &   3.581852 \\
Rast n inter plus  tau=4 tau prima=1       &  0.00 &  250.0 &   4.118031 \\
Rast n inter plus  tau=4 tau prima=0.5     &  0.00 &  250.0 &   8.075069 \\
Rast n inter plus  tau=4 tau prima=0.25    &  0.00 &  250.0 &  18.080988 \\
Rast n inter plus  tau=2 tau prima=4       &  0.00 &  250.0 &   5.124038 \\
Rast n inter plus  tau=2 tau prima=2       &  0.00 &  250.0 &   4.278324 \\
Rast n inter plus  tau=2 tau prima=0.5     &  0.00 &  250.0 &  16.547862 \\
Rast n inter plus  tau=2 tau prima=0.25    &  0.00 &  250.0 &  26.946989 \\
Rast n inter plus  tau=1 tau prima=4       &  0.00 &  250.0 &   4.925047 \\
Rast n inter plus  tau=1 tau prima=1       &  0.00 &  250.0 &   4.427567 \\
Rast n inter plus  tau=1 tau prima=0.5     &  0.00 &  250.0 &  20.888121 \\
Rast n inter plus  tau=1 tau prima=0.25    &  0.00 &  250.0 &  27.522825 \\
Rast n inter plus  tau=0.5 tau prima=1     &  0.00 &  250.0 &   3.681348 \\
Rast n inter plus  tau=0.5 tau prima=0.5   &  0.00 &  250.0 &  22.831933 \\
Rast n inter plus  tau=0.5 tau prima=0.25  &  0.00 &  250.0 &  25.594012 \\
Rast n inter plus  tau=0.25 tau prima=4    &  0.00 &  250.0 &   4.328071 \\
Rast n inter plus  tau=0.25 tau prima=2    &  0.00 &  250.0 &   4.129080 \\
Rast n disc  plus  tau=0.5 tau prima=0.25  &  0.00 &  250.0 &  25.855713 \\
Rast n disc  plus  tau=0.5 tau prima=0.5   &  0.00 &  250.0 &   4.278323 \\
Rast n disc  plus  tau=0.5 tau prima=1     &  0.00 &  250.0 &   4.328071 \\
Rast n disc  plus  tau=0.5 tau prima=2     &  0.00 &  250.0 &   4.726054 \\
Rast one disc  plus  tau=1                 &  0.00 &  250.0 &   2.984877 \\
Rast one disc  plus  tau=0.25              &  0.00 &  250.0 &  27.031063 \\
Rast one inter plus  tau=4                 &  0.00 &  250.0 &   4.129080 \\
Rast one inter plus  tau=0.5               &  0.00 &  250.0 &  23.401600 \\
Rast one inter plus  tau=0.25              &  0.00 &  250.0 &  25.288749 \\
Rast one disc  comma tau=4                 &  0.00 &  250.0 &   5.024542 \\
Rast one disc  comma tau=2                 &  0.00 &  250.0 &   2.935129 \\
Rast one disc  comma tau=1                 &  0.00 &  250.0 &   3.183867 \\
Rast one inter comma tau=2                 &  0.00 &  250.0 &   2.736137 \\
Rast n disc  plus  tau=4 tau prima=4       &  0.00 &  250.0 &   4.136850 \\
Rast n inter plus  tau=0.25 tau prima=0.5  &  0.00 &  250.0 &  22.236333 \\
Rast n disc  plus  tau=4 tau prima=0.5     &  0.00 &  250.0 &   4.228575 \\
Rast n disc  plus  tau=2 tau prima=2       &  0.00 &  250.0 &   5.522023 \\
Rast n disc  plus  tau=2 tau prima=1       &  0.00 &  250.0 &   3.979835 \\
Rast n disc  plus  tau=2 tau prima=0.5     &  0.00 &  250.0 &   5.770760 \\
Rast n disc  plus  tau=2 tau prima=0.25    &  0.00 &  250.0 &   4.974795 \\
Rast n disc  plus  tau=1 tau prima=4       &  0.00 &  250.0 &   4.990099 \\
Rast n disc  plus  tau=1 tau prima=2       &  0.00 &  250.0 &   5.671265 \\
Rast n disc  plus  tau=1 tau prima=1       &  0.00 &  250.0 &   3.581853 \\
Rast n disc  plus  tau=1 tau prima=0.5     &  0.00 &  250.0 &   4.377818 \\
Rast n disc  plus  tau=1 tau prima=0.25    &  0.00 &  250.0 &   6.613558 \\
Rast n disc  plus  tau=0.5 tau prima=4     &  0.00 &  250.0 &   4.463781 \\
Rast n disc  plus  tau=2 tau prima=4       &  0.00 &  250.0 &   4.957643 \\
Rast n inter plus  tau=0.25 tau prima=0.25 &  0.00 &  250.0 &  24.736600 \\
Rast one disc  plus  tau=4                 &  0.00 &  250.0 &   4.676306 \\
Rast n disc  comma tau=4 tau prima=2       &  0.00 &  250.0 &   5.323029 \\
Rast n disc  comma tau=0.5 tau prima=0.25  &  0.00 &  250.0 &   4.029583 \\
Rast n disc  comma tau=0.25 tau prima=1    &  0.00 &  250.0 &   3.134120 \\
Rast n disc  comma tau=0.25 tau prima=0.25 &  0.00 &  250.0 &   4.029584 \\
Rast n inter comma tau=4 tau prima=2       &  0.00 &  250.0 &   3.183869 \\
Rast n disc  comma tau=4 tau prima=4       &  0.00 &  250.0 &   4.620339 \\
Rast n inter comma tau=4 tau prima=0.25    &  0.00 &  250.0 &  69.340123 \\
Rast n inter comma tau=2 tau prima=4       &  0.00 &  250.0 &   3.979836 \\
Rast n disc  comma tau=0.5 tau prima=1     &  0.00 &  250.0 &   3.830591 \\
Rast n inter comma tau=2 tau prima=2       &  0.00 &  250.0 &   2.885381 \\
Rast n inter comma tau=1 tau prima=4       &  0.00 &  250.0 &   3.532104 \\
Rast n inter comma tau=1 tau prima=2       &  0.00 &  250.0 &   2.785885 \\
Rast n inter comma tau=1 tau prima=0.25    &  0.00 &  250.0 &  38.389069 \\
Rast n inter comma tau=0.5 tau prima=4     &  0.00 &  250.0 &   3.930088 \\
Rast n inter comma tau=0.5 tau prima=2     &  0.00 &  250.0 &   3.034625 \\
Rast n inter comma tau=0.25 tau prima=4    &  0.00 &  250.0 &   4.079331 \\
Rast n inter comma tau=0.25 tau prima=2    &  0.00 &  250.0 &   2.785885 \\
Rast n inter comma tau=2 tau prima=0.25    &  0.00 &  250.0 &  60.468894 \\
Rast n disc  comma tau=0.5 tau prima=2     &  0.00 &  250.0 &   4.626559 \\
Rast n inter comma tau=4 tau prima=0.5     &  0.00 &  250.0 &  56.879541 \\
Rast n disc  comma tau=1 tau prima=0.25    &  0.00 &  250.0 &   3.830592 \\
Rast n disc  comma tau=2 tau prima=4       &  0.00 &  250.0 &   4.186123 \\
Rast n disc  comma tau=2 tau prima=1       &  0.00 &  250.0 &   3.731096 \\
Rast n disc  comma tau=2 tau prima=0.5     &  0.00 &  250.0 &   3.731096 \\
Rast n disc  comma tau=2 tau prima=0.25    &  0.00 &  250.0 &   4.228575 \\
Rast n disc  comma tau=4 tau prima=0.25    &  0.00 &  250.0 &   4.278323 \\
Rast n disc  comma tau=1 tau prima=4       &  0.00 &  250.0 &   5.176250 \\
Rast n disc  comma tau=0.5 tau prima=4     &  0.00 &  250.0 &   4.242211 \\
Rast n disc  comma tau=1 tau prima=2       &  0.00 &  250.0 &   4.079332 \\
Rast n disc  comma tau=2 tau prima=2       &  0.00 &  250.0 &   3.631600 \\
Rast n disc  comma tau=4 tau prima=1       &  0.00 &  250.0 &   4.576810 \\
Rast n disc  comma tau=1 tau prima=1       &  0.00 &  250.0 &   3.233616 \\
Rast n disc  comma tau=1 tau prima=0.5     &  0.00 &  250.0 &   4.477315 \\
Rast n disc  comma tau=4 tau prima=0.5     &  0.00 &  250.0 &   4.676307 \\
Rast one inter plus  tau=1                 &  0.05 &  250.0 &   4.911691 \\
Rast one inter comma tau=4                 &  0.05 &  250.0 &   3.681348 \\
Rast one inter plus  tau=2                 &  0.05 &  250.0 &   3.034625 \\
Rast n inter plus  tau=0.25 tau prima=1    &  0.05 &  250.0 &   2.984877 \\
Rast n inter comma tau=0.25 tau prima=1    &  0.05 &  250.0 &   2.139162 \\
Rast one disc  plus  tau=0.5               &  0.05 &  250.0 &   2.785885 \\
Rast one inter comma tau=0.25              &  0.05 &  250.0 &  28.578686 \\
Rast n disc  plus  tau=4 tau prima=2       &  0.05 &  250.0 &   3.432608 \\
Rast one disc  comma tau=0.25              &  0.05 &  250.0 &   2.039666 \\
Rast n disc  plus  tau=4 tau prima=0.25    &  0.05 &  250.0 &   4.228575 \\
Rast n inter plus  tau=0.5 tau prima=2     &  0.05 &  250.0 &   4.328071 \\
Rast n disc  comma tau=0.5 tau prima=0.5   &  0.05 &  250.0 &   3.432609 \\
Rast n inter plus  tau=2 tau prima=1       &  0.05 &  250.0 &   3.681348 \\
Rast n disc  comma tau=0.25 tau prima=4    &  0.05 &  250.0 &   3.930058 \\
Rast n disc  comma tau=0.25 tau prima=2    &  0.05 &  250.0 &   3.681348 \\
Rast n disc  plus  tau=0.25 tau prima=4    &  0.05 &  250.0 &   4.525109 \\
Rast n inter comma tau=4 tau prima=4       &  0.05 &  250.0 &   3.034625 \\
Rast n inter comma tau=4 tau prima=1       &  0.05 &  250.0 &   2.139162 \\
Rast n disc  comma tau=0.25 tau prima=0.5  &  0.05 &  250.0 &   3.532104 \\
Rast n inter comma tau=2 tau prima=1       &  0.05 &  250.0 &   2.387902 \\
Rast one disc  plus  tau=2                 &  0.05 &  250.0 &   3.532104 \\
Rast n inter plus  tau=1 tau prima=2       &  0.05 &  250.0 &   4.477315 \\
Rast one inter comma tau=1                 &  0.10 &  250.0 &   1.741178 \\
Rast n disc  plus  tau=0.25 tau prima=0.5  &  0.10 &  250.0 &   4.029583 \\
Rast n disc  plus  tau=4 tau prima=1       &  0.10 &  250.0 &   3.780844 \\
Rast n inter comma tau=1 tau prima=0.5     &  0.10 &  250.0 &   1.492439 \\
Rast n inter comma tau=0.5 tau prima=1     &  0.15 &  250.0 &   2.288406 \\
Rast n inter comma tau=1 tau prima=1       &  0.20 &  250.0 &   1.989918 \\
Rast n inter comma tau=0.25 tau prima=0.5  &  0.20 &  250.0 &   1.392943 \\
Rast one disc  comma tau=0.5               &  0.25 &  250.0 &   2.288404 \\
Rast n inter comma tau=0.5 tau prima=0.5   &  0.25 &  250.0 &   1.343195 \\
Rast n inter comma tau=0.25 tau prima=0.25 &  0.25 &  250.0 &   5.263288 \\
Rast one inter comma tau=0.5               &  0.30 &  250.0 &   1.094465 \\
Rast n inter comma tau=0.5 tau prima=0.25  &  0.30 &  250.0 &  10.345627 \\
Rast n inter comma tau=2 tau prima=0.5     &  0.40 &  250.0 &   0.696471 \\
\bottomrule
\end{tabular}

    

    \section{Resultados, análisis y
comparación}\label{resultados-anuxe1lisis-y-comparaciuxf3n}

Los resultados los he ido poniendo en las distintas secciones de esta
práctica. También los he ido comentando.

Sin embargo, aquí me interesaría proceder a un análisis del ataque por
fuerza bruta anterior para determinar la mejor configuración de la
estrategia para la función de Rastrigin.

Podemos concluir, de la tabla anterior:

\begin{itemize}
\item
  El paso n suele ir mejor que el sencillo
\item
  La selección mu\_comma\_lambda sin ninguna duda van mejor que la
  mu\_plus\_lambda
\item
  La recombinación intermedia ofrece mejores resultados que la discreta
\item
  El tau más apropiado parece situarse en algún punto entre 0.25 y 0.5
  aunque curiosamente el mejor resultado se obtiene con un tau igual a 2
\item
  Los mejores taus primas parece que están también en algún punto entre
  0.25 y 0.5
\end{itemize}

Y en cualquier caso, y aunque contraviene un poco las observaciones
anteriores, la mejor configuracion parece que es paso n, selección
mu\_comma\_lambda, recombinación intermediate, tau=2 y tau prima=0.5.

    \section{Conclusiones}\label{conclusiones}

Después de todo el trabajo que lleva esta práctica llego a la conclusión
de que las estrategias evolutivas pueden ser una buena herramiento para
búsqueda de óptimos en problemas de naturaleza numérica o que puedan
modelarse a esa naturaleza.

La cantidad de distintas parametrizaciones que pueden hacerse en estas
estrategias es elevada y dar con un algoritmo útil para cada problema
puede conllevar mucho más trabajo de lo que en principio aparenta.

Es muy importante comprender el problema real que se quiere resolver
para conducirse por el camino de parametrizar apropiadamente el
algoritmo. Por ejemplo, en el caso de la función de Rastrigin ha sido
muy complicado comprender por qué la mayoría de las veces el algoritmo
se quedaba ``atrapado'' en un óptimo cercano al origen de coordenadas.
Tras dibujar la función en dos dimensiones es más fácil comprender que
realmente lo que ocurre en vista de la gran cantidad de óptimos locales
que tiene la función y su escasa diferencia en valor.Si se tiene en
cuenta que el problema está planteado en 10 dimensiones, puedo
comprender que los resultados no siempre hayan sido buenos. En el caso
de 6 dimensiones, las pruebas que he hecho siempre encuentran el óptimo
global. Y en el caso de 8 dimensiones, la mayoría de las veces lo
encuentran.

Tengo la sensación de no haber conseguido exprimir al máximo las
posibilidades que dan los parámetros tau y tau prima, y no será porque
no he hecho miles de pruebas. Pero en cualquier caso me quedo con la
idea de que gobirnar la intensidad de la variabilidad de las sigmas.

En el caso de la función hiperelipsoide rotado, el tener un único
óptimo, todas las configuraciones del algoritmo han dado resultados muy
buenos. Nuevamente, esto se comprenden muy fácilmente representando la
función en el caso de dos dimensiones .

Lo realmente importante, o lo que más importante me ha resultado a mí de
esta práctica, es recabar en la necesidad de utilizar métricas de
estimación de la calidad de un algoritmo y su parametrización y que
estas métricas se basen en la repetición de ejecucuiones idénticas y en
el análisis estadístico de los resultados. De este modo, no podria haber
hecho comparaciones que realmente valgan para algo sin haber dispuesto
de las tres métricas SR, AES y MBF.

Igualmente, dibujar gráficas de la evolución del fitness considero que
es de lo más necesario. Me ha sido muy útil acudir al concepto de escala
logarítmica (inversa en este caso), para que las gráficas realmente
tubieran sentido y fueran interpretables.


    % Add a bibliography block to the postdoc
    
    
    
    \end{document}
